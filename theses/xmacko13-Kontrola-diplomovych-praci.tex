%==============================================================================
% Tento soubor použijte jako základ
% This file should be used as a base for the thesis
% Autoři / Authors: 2008 Michal Bidlo, 2022 Jaroslav Dytrych
% Kontakt pro dotazy a připomínky: sablona@fit.vutbr.cz
% Contact for questions and comments: sablona@fit.vutbr.cz
%==============================================================================
% kódování: UTF-8 (zmena prikazem iconv, recode nebo cstocs)
% encoding: UTF-8 (you can change it by command iconv, recode or cstocs)
%------------------------------------------------------------------------------
% zpracování / processing: make, make pdf, make clean
%==============================================================================
% Soubory, které je nutné upravit nebo smazat: / Files which have to be edited or deleted:
%   xmacko13-Kontrola-diplomovych-praci-20-literatura-bibliography.bib - literatura / bibliography
%   xmacko13-Kontrola-diplomovych-praci-01-kapitoly-chapters.tex - obsah práce / the thesis content
%   xmacko13-Kontrola-diplomovych-praci-01-kapitoly-chapters-en.tex - obsah práce v angličtině / the thesis content in English
%   xmacko13-Kontrola-diplomovych-praci-30-prilohy-appendices.tex - přílohy / appendices
%   xmacko13-Kontrola-diplomovych-praci-30-prilohy-appendices-en.tex - přílohy v angličtině / appendices in English
%==============================================================================
%\documentclass[]{fitthesis} % bez zadání - pro začátek práce, aby nebyl problém s překladem
%\documentclass[english]{fitthesis} % without assignment - for the work start to avoid compilation problem
\documentclass[zadani]{fitthesis} % odevzdani do IS VUT a/nebo tisk s barevnými odkazy - odkazy jsou barevné
%\documentclass[english,zadani]{fitthesis} % for submission to the IS VUT and/or print with color links - links are color
%\documentclass[zadani,print]{fitthesis} % pro černobílý tisk - odkazy jsou černé
%\documentclass[english,zadani,print]{fitthesis} % for the black and white print - links are black
%\documentclass[zadani,cprint]{fitthesis} % pro barevný tisk - odkazy jsou černé, znak VUT barevný
%\documentclass[english,zadani,cprint]{fitthesis} % for the print - links are black, logo is color
% * Je-li práce psaná v anglickém jazyce, je zapotřebí u třídy použít 
%   parametr english následovně:
%   If thesis is written in English, it is necessary to use 
%   parameter english as follows:
%      \documentclass[english]{fitthesis}
% * Je-li práce psaná ve slovenském jazyce, je zapotřebí u třídy použít 
%   parametr slovak následovně:
%   If the work is written in the Slovak language, it is necessary 
%   to use parameter slovak as follows:
%      \documentclass[slovak]{fitthesis}
% * Je-li práce psaná v anglickém jazyce se slovenským abstraktem apod., 
%   je zapotřebí u třídy použít parametry english a enslovak následovně:
%   If the work is written in English with the Slovak abstract, etc., 
%   it is necessary to use parameters english and enslovak as follows:
%      \documentclass[english,enslovak]{fitthesis}


% Základní balíčky jsou dole v souboru šablony fitthesis.cls
% Basic packages are at the bottom of template file fitthesis.cls
% zde můžeme vložit vlastní balíčky / you can place own packages here


% Pro seznam zkratek lze využít balíček Glossaries - nutno odkomentovat i níže a při kompilaci z konzoly i v Makefile (plnou verzi pro Perl, nebo lite)
% The Glossaries package can be used for the list of abbreviations - it is necessary to uncomment also below. When compiling from the console also in the Makefile (full version for Perl or lite)
%\usepackage{glossaries}
%\usepackage{glossary-superragged}
%\makeglossaries 

% Nastavení cesty k obrázkům
% Setting of a path to the pictures
%\graphicspath{{obrazky-figures/}{./obrazky-figures/}}
%\graphicspath{{obrazky-figures/}{../obrazky-figures/}}

%---rm---------------
\renewcommand{\rmdefault}{lmr}%zavede Latin Modern Roman jako rm / set Latin Modern Roman as rm
%---sf---------------
\renewcommand{\sfdefault}{qhv}%zavede TeX Gyre Heros jako sf
%---tt------------
\renewcommand{\ttdefault}{lmtt}% zavede Latin Modern tt jako tt

% vypne funkci šablony, která automaticky nahrazuje uvozovky,
% aby nebyly prováděny nevhodné náhrady v popisech API apod.
% disables function of the template which replaces quotation marks
% to avoid unnecessary replacements in the API descriptions etc.
\csdoublequotesoff

\usepackage{url}
\usepackage{xcolor}
\newcommand{\dummyText}[1][1]{{\color{lightgray}\blindtext[#1]}}
\newcommand{\DummyText}[0]{{\color{lightgray}\Blindtext}}
\usepackage{pgffor}
\newcommand{\blindshort}[1][3]{
  \def\loremtext{
    {Lorem ipsum dolor sit amet, consectetuer adipiscing elit.},
    {Etiam lobortis facilisis sem.},
    {Nullam nec mi et neque pharetra sollicitudin.},
    {Praesent imperdiet mi nec ante.},
    {Donec ullamcorper, felis non sodales commodo, lectus velit ultrices augue, a dignissim nibh lectus placerat pede.},
    {Vivamus nunc nunc, molestie ut, ultricies vel, semper in, velit.},
    {Ut porttitor.},
    {Praesent in sapien.},
    {Lorem ipsum dolor sit amet, consectetuer adipiscing elit.},
    {Duis fringilla tristique neque.},
    {Sed interdum libero ut metus.},
    {Pellentesque placerat.},
    {Nam rutrum augue a leo.},
    {Morbi sed elit sit amet ante lobortis sollicitudin.},
    {Praesent blandit blandit mauris.},
    {Praesent lectus tellus, aliquet aliquam, luctus a, egestas a, turpis.},
    {Mauris lacinia lorem sit amet ipsum.},
    {Nunc quis urna dictum turpis accumsan semper.}}%
  \foreach \texti [count=\i] in \loremtext {
    \ifnum\i>#1
    \else
    \texti
    \fi
  }
}
\newcommand{\dummyShortText}[1][3]{{\color{lightgray}\blindshort[#1]}}

%listing utf-8 chars (czech + few more)
\lstset{
  literate=%
    {á}{{\'a}}1
    {í}{{\'i}}1
    {é}{{\'e}}1
    {ý}{{\'y}}1
    {ú}{{\'u}}1
    {ó}{{\'o}}1
    {ě}{{\v{e}}}1
    {š}{{\v{s}}}1
    {č}{{\v{c}}}1
    {ř}{{\v{r}}}1
    {ž}{{\v{z}}}1
    {ď}{{\v{d}}}1
    {ť}{{\v{t}}}1
    {ň}{{\v{n}}}1                
    {ů}{{\r{u}}}1
    {Á}{{\'A}}1
    {Í}{{\'I}}1
    {É}{{\'E}}1
    {Ý}{{\'Y}}1
    {Ú}{{\'U}}1
    {Ó}{{\'O}}1
    {Ě}{{\v{E}}}1
    {Š}{{\v{S}}}1
    {Č}{{\v{C}}}1
    {Ř}{{\v{R}}}1
    {Ž}{{\v{Z}}}1
    {Ď}{{\v{D}}}1
    {Ť}{{\v{T}}}1
    {Ň}{{\v{N}}}1                
    {Ů}{{\r{U}}}1
    {ŕ}{{\'r}}1
    {ś}{{\'s}}1
}

\lstdefinestyle{myPython}{
  language=Python,
  basicstyle=\fontfamily{newtxtt}\small,
  numbers=left,
  numberstyle=\scriptsize,
  frame=lines,
  xleftmargin=2em,
  xrightmargin=0.5em,
  %framexleftmargin=1.5em,
  captionpos=b,
}

% =======================================================================
% balíček "hyperref" vytváří klikací odkazy v pdf, pokud tedy použijeme pdflatex
% problém je, že balíček hyperref musí být uveden jako poslední, takže nemůže
% být v šabloně
% "hyperref" package create clickable links in pdf if you are using pdflatex.
% Problem is that this package have to be introduced as the last one so it 
% can not be placed in the template file.
\ifWis
\ifx\pdfoutput\undefined % nejedeme pod pdflatexem / we are not using pdflatex
\else
  \usepackage{color}
  \usepackage[unicode,colorlinks,hyperindex,plainpages=false,pdftex]{hyperref}
  \definecolor{hrcolor-ref}{RGB}{223,52,30}
  \definecolor{hrcolor-cite}{HTML}{2F8F00}
  \definecolor{hrcolor-urls}{HTML}{092EAB}
  \hypersetup{
	linkcolor=hrcolor-ref,
	citecolor=hrcolor-cite,
	filecolor=magenta,
	urlcolor=hrcolor-urls
  }
  \def\pdfBorderAttrs{/Border [0 0 0] }  % bez okrajů kolem odkazů / without margins around links
  \pdfcompresslevel=9
\fi
\else % pro tisk budou odkazy, na které se dá klikat, černé / for the print clickable links will be black
\ifx\pdfoutput\undefined % nejedeme pod pdflatexem / we are not using pdflatex
\else
  \usepackage{color}
  \usepackage[unicode,colorlinks,hyperindex,plainpages=false,pdftex,urlcolor=black,linkcolor=black,citecolor=black]{hyperref}
  \definecolor{links}{rgb}{0,0,0}
  \definecolor{anchors}{rgb}{0,0,0}
  \def\AnchorColor{anchors}
  \def\LinkColor{links}
  \def\pdfBorderAttrs{/Border [0 0 0] } % bez okrajů kolem odkazů / without margins around links
  \pdfcompresslevel=9
\fi
\fi
% Řešení problému, kdy klikací odkazy na obrázky vedou za obrázek
% This solves the problems with links which leads after the picture
\usepackage[all]{hypcap}


% Informace o práci/projektu / Information about the thesis
%---------------------------------------------------------------------------
\projectinfo{
  %Prace / Thesis
  project={BP},            %typ práce BP/SP/DP/DR  / thesis type (SP = term project)
  year={2023},             % rok odevzdání / year of submission
  date=\today,             % datum odevzdání / submission date
  %Nazev prace / thesis title
  title.cs={Nástroj pro kontrolu diplomových prací},  % název práce v češtině či slovenštině (dle zadání) / thesis title in czech language (according to assignment)
  title.en={Theses Checker}, % název práce v angličtině / thesis title in english
  %title.length={14.5cm}, % nastavení délky bloku s titulkem pro úpravu zalomení řádku (lze definovat zde nebo níže) / setting the length of a block with a thesis title for adjusting a line break (can be defined here or below)
  %sectitle.length={14.5cm}, % nastavení délky bloku s druhým titulkem pro úpravu zalomení řádku (lze definovat zde nebo níže) / setting the length of a block with a second thesis title for adjusting a line break (can be defined here or below)
  %dectitle.length={14.5cm}, % nastavení délky bloku s titulkem nad prohlášením pro úpravu zalomení řádku (lze definovat zde nebo níže) / setting the length of a block with a thesis title above declaration for adjusting a line break (can be defined here or below)
  %Autor / Author
  author.name={Michaela},   % jméno autora / author name
  author.surname={Macková},   % příjmení autora / author surname 
  %author.title.p={Bc.}, % titul před jménem (nepovinné) / title before the name (optional)
  %author.title.a={Ph.D.}, % titul za jménem (nepovinné) / title after the name (optional)
  %Ustav / Department
  department={UPGM}, % doplňte příslušnou zkratku dle ústavu na zadání: UPSY/UIFS/UITS/UPGM / fill in appropriate abbreviation of the department according to assignment: UPSY/UIFS/UITS/UPGM
  % Školitel / supervisor
  supervisor.name={Tomáš},   % jméno školitele / supervisor name 
  supervisor.surname={Milet},   % příjmení školitele / supervisor surname
  supervisor.title.p={Ing.},   %titul před jménem (nepovinné) / title before the name (optional)
  supervisor.title.a={Ph.D.},    %titul za jménem (nepovinné) / title after the name (optional)
  % Klíčová slova / keywords
  keywords.cs={Sem budou zapsána jednotlivá klíčová slova v českém (slovenském) jazyce, oddělená čárkami.}, % klíčová slova v českém či slovenském jazyce / keywords in czech or slovak language
  keywords.en={Sem budou zapsána jednotlivá klíčová slova v anglickém jazyce, oddělená čárkami.}, % klíčová slova v anglickém jazyce / keywords in english
  %keywords.en={Here, individual keywords separated by commas will be written in English.},
  % Abstrakt / Abstract
  abstract.cs={Do tohoto odstavce bude zapsán výtah (abstrakt) práce v českém (slovenském) jazyce.}, % abstrakt v českém či slovenském jazyce / abstract in czech or slovak language
  abstract.en={Do tohoto odstavce bude zapsán výtah (abstrakt) práce v anglickém jazyce.}, % abstrakt v anglickém jazyce / abstract in english
  %abstract.en={An abstract of the work in English will be written in this paragraph.},
  % Prohlášení (u anglicky psané práce anglicky, u slovensky psané práce slovensky; u projektové praxe lze zakomentovat) / Declaration (for thesis in english should be in english; for project practice can be commented out)
  declaration={Prohlašuji, že jsem tuto bakalářskou práci vypracovala samostatně pod vedením pana Ing.~Tomáše Mileta, Ph.D.
Další informace mi poskytli... \todo{TODO}
Uvedla jsem všechny literární prameny, publikace a~další zdroje, ze kterých jsem čerpala.},
  %declaration={I hereby declare that this Bachelor's thesis was prepared as an original work by the author under the supervision of Mr. X
% The supplementary information was provided by Mr. Y
% I have listed all the literary sources, publications and other sources, which were used during the preparation of this thesis.},
  % Poděkování (nepovinné, nejlépe v jazyce práce; nechcete-li, zakomentujte pro skrytí nadpisu) / Acknowledgement (optional, ideally in the language of the thesis; comment out for hiding including heading)
  acknowledgment={V této sekci je možno uvést poděkování vedoucímu práce a těm, kteří poskytli odbornou pomoc
(externí zadavatel, konzultant apod.).},
  %acknowledgment={Here it is possible to express thanks to the supervisor and to the people which provided professional help
%(external submitter, consultant, etc.).},
  % Rozšířený abstrakt (cca 3 normostrany) - lze definovat zde nebo níže / Extended abstract (approximately 3 standard pages) - can be defined here or below
  %extendedabstract={Do tohoto odstavce bude zapsán rozšířený výtah (abstrakt) práce v českém (slovenském) jazyce.},
  %extabstract.odd={true}, % Začít rozšířený abstrakt na liché stránce? / Should extended abstract start on the odd page?
  %faculty={FIT}, % FIT/FEKT/FSI/FA/FCH/FP/FAST/FAVU/USI/DEF
  faculty.cs={Fakulta informačních technologií}, % Fakulta v češtině - pro využití této položky výše zvolte fakultu DEF / Faculty in Czech - for use of this entry select DEF above
  faculty.en={Faculty of Information Technology}, % Fakulta v angličtině - pro využití této položky výše zvolte fakultu DEF / Faculty in English - for use of this entry select DEF above
  department.cs={Ústav matematiky}, % Ústav v češtině - pro využití této položky výše zvolte ústav DEF nebo jej zakomentujte / Department in Czech - for use of this entry select DEF above or comment it out
  department.en={Institute of Mathematics} % Ústav v angličtině - pro využití této položky výše zvolte ústav DEF nebo jej zakomentujte / Department in English - for use of this entry select DEF above or comment it out
}

% Rozšířený abstrakt (cca 3 normostrany) - lze definovat zde nebo výše / Extended abstract (approximately 3 standard pages) - can be defined here or above
%\extendedabstract{Do tohoto odstavce bude zapsán výtah (abstrakt) práce v českém (slovenském) jazyce.}
% Začít rozšířený abstrakt na liché stránce? / Should extended abstract start on the odd page?
%\extabstractodd{true}

% nastavení délky bloku s titulkem pro úpravu zalomení řádku - lze definovat zde nebo výše / setting the length of a block with a thesis title for adjusting a line break - can be defined here or above
%\titlelength{14.5cm}
% nastavení délky bloku s druhým titulkem pro úpravu zalomení řádku - lze definovat zde nebo výše / setting the length of a block with a second thesis title for adjusting a line break - can be defined here or above
%\sectitlelength{14.5cm}
% nastavení délky bloku s titulkem nad prohlášením pro úpravu zalomení řádku - lze definovat zde nebo výše / setting the length of a block with a thesis title above declaration for adjusting a line break - can be defined here or above
%\dectitlelength{14.5cm}

% řeší první/poslední řádek odstavce na předchozí/následující stránce
% solves first/last row of the paragraph on the previous/next page
\clubpenalty=10000
\widowpenalty=10000

% checklist
\newlist{checklist}{itemize}{1}
\setlist[checklist]{label=$\square$}

% Kompilace po částech (rychlejší, ale v náhledu nemusí být vše aktuální)
% Compilation piecewise (faster, but not all parts in preview will be up-to-date)
% Další informace viz / For more information see https://www.overleaf.com/learn/latex/Multi-file_LaTeX_projects
% \usepackage{subfiles}

% Nechcete-li, aby se u oboustranného tisku roztahovaly mezery pro zaplnění stránky, odkomentujte následující řádek / If you do not want enlarged spacing for filling of the pages in case of duplex printing, uncomment the following line
% \raggedbottom

\begin{document}
  % Vysazeni titulnich stran / Typesetting of the title pages
  % ----------------------------------------------
  \maketitle
  % Obsah
  % ----------------------------------------------
  \setlength{\parskip}{0pt}
  \setcounter{tocdepth}{1}
  {\hypersetup{hidelinks}\tableofcontents}
  
  % Seznam obrazku a tabulek (pokud prace obsahuje velke mnozstvi obrazku, tak se to hodi)
  % List of figures and list of tables (if the thesis contains a lot of pictures, it is good)
  \ifczech
    \renewcommand\listfigurename{Seznam obrázků}
  \fi
  \ifslovak
    \renewcommand\listfigurename{Zoznam obrázkov}
  \fi
  %{\hypersetup{hidelinks}\listoffigures} % TODO: zakomentovat?
  
  \ifczech
    \renewcommand\listtablename{Seznam tabulek}
  \fi
  \ifslovak
    \renewcommand\listtablename{Zoznam tabuliek}
  \fi
  % {\hypersetup{hidelinks}\listoftables}

  % Seznam zkratek / List of abbreviations
  %\ifczech
  %  \renewcommand*\glossaryname{Seznam zkratek}%
  %  \renewcommand*\entryname{Zkratka}
  %  \renewcommand*\descriptionname{Význam}
  %\fi
  %\ifslovak
  %  \renewcommand*\glossaryname{Zoznam skratiek}%
  %  \renewcommand*\entryname{Skratka}
  %  \renewcommand*\descriptionname{Význam}
  %\fi
  %\ifenglish
  %  \renewcommand*\glossaryname{List of abbreviations}%
  %  \renewcommand*\entryname{Abbreviation}
  %  \renewcommand*\descriptionname{Meaning}
  %\fi
  % Definice zkratek - z textu se odkazují např. \Gls{TF–IDF}
  % Definition of abbreviations - referred from the text e.g. \Gls{TF–IDF}
  %\newglossaryentry{TF–IDF}
  %{
  %  name={TF–IDF},
  %  description={Term Frequency-Inverse Document Frequency}
  %}
  % 
  %\setglossarystyle{superragged}
  %\printglossaries


  \ifODSAZ
    \setlength{\parskip}{0.5\bigskipamount}
  \else
    \setlength{\parskip}{0pt}
  \fi

  % vynechani stranky v oboustrannem rezimu
  % Skip the page in the two-sided mode
  \iftwoside
    \cleardoublepage
  \fi

  % Text prace / Thesis text
  % ----------------------------------------------
  \ifenglish
    \input{xmacko13-Kontrola-diplomovych-praci-01-kapitoly-chapters-en}
  \else
    % Tento soubor nahraďte vlastním souborem s obsahem práce.
%=========================================================================
% Autoři: Michal Bidlo, Bohuslav Křena, Jaroslav Dytrych, Petr Veigend a Adam Herout 2019

% Pro kompilaci po částech (viz projekt.tex), nutno odkomentovat a upravit
%\documentclass[../projekt.tex]{subfiles}
%\begin{document}


\newcommand{\highlight}[1]{\colorbox{purple}{\color{white}#1}}
\newcommand{\todoimage}[2]{
    \begin{figure}[H]
        \centering
        \includegraphics[#1]{obrazky-figures/placeholder.pdf}
        \caption{\textbf{#2} \todo{popisek}}
    \end{figure}
}


% \renewcommand{\dummyShortText}[1][1]{}
% \renewcommand{\dummyText}[1][1]{}
% \renewcommand{\DummyText}{}
% \renewcommand{\todoimage}[2]{}


%*********************************************************************************
%                                    1 ÚVOD
%*********************************************************************************
\chapter{Úvod}

\dummyText

\dummyText[2]

\dummyShortText[15]

%*********************************************************************************




%*********************************************************************************
%                                 2 TYPOGRAFIE
%*********************************************************************************
\chapter{Rychlokurz typografie}


%*********************************************************************************




%*********************************************************************************
%                                3 ČASTÉ CHYBY
%*********************************************************************************
\chapter{Často vyskytované chyby v~diplomových pracích}
U~psaní textu se autor musí řídit ne jen gramatickými, ale i~typografickými
pravidly. Toto především platí při psaní odborné práce. Větší množství chyb
v~obsahu práce může mít za následek, že i~kvalitně odvedená práce se bude zdát
neuspokojivá.

Chyby mohou být způsobeny z~nepozornosti, nebo z~neznalosti, přičemž druhá možnost
je pro autora textu horší, jelikož i~po několikátém přečtení nemusí pisatel vůbec
poznat, že se jedná o~chybu. Správnou volbou textového editoru si tvůrce textu
může usnadnit hledání některých chyb. Některé dnešní textové procesory poskytují
alespoň částečnou kontrolu pravopisu, nicméně tato kontrola umí ve spoustě případů
upozornit převážně jen na překlepy. Významové chyby, jako je například záměna slov
\emph{tip, typ} nebo \emph{autorizace, autentizace}, bývají často touto
automatickou kontrolou zanedbávány. Další části této kapitoly popisují několik
chyb, které lze nalézt v~mnoha diplomových pracích a~dále uvádějí jak se těmto
chybám vyhnout.


%#######################    3.1 Přetečení obsahu za okraj    #######################
\section{Přetečení obsahu za okraj}
Přetečení textu za okraj se nejčastěji vyskytuje, když student píše svou diplomovou
práci s~pomocí jazyka {\LaTeX}. Obvykle je to způsobeno tím, že program nedokáže
automaticky zalomit slovo na konci řádku, jak je ukázáno na
obrázku~\ref{pic_overflow}. Toto lze opravit napověděním možného
zalomení nebo přeformulováním věty, kde se daná chyba vyskytuje.
Další typ této chyby je přetečení obrázku za okraj,
který se nestává tak často, ale lze jej udělat v~několika textových editorech.

\begin{figure}[H]
    \label{pic_overflow}
    \centering
    %\includegraphics[width=\linewidth,height=1.7in]{obrazky-figures/placeholder.pdf}
    \includegraphics{obrazky-figures/overflow.pdf}
    \caption{\textbf{Ukázka přetečení za okraj.} \todo{popisek}}
\end{figure}


%#######################    3.2 Spojovník x pomlčka    #######################
\section{Špatné použití spojovníku}
Špatné používání spojovníku je chyba, která se vyskytuje nejen v~diplomových
pracích. Spojovník (-) je graficky velmi podobný pomlčce (--), ale významově
se značně liší. Pravidla pro psaní těchto znaků, uvedena v~internetové příručce
Ústavu pro jazyk český~\cite{Ustav_pro_jazyk_cesky},
říkají, že spojovník se píše bez mezer mezi výrazy, které spojuje. Výjimkou
je, naznačuje-li spojovník neúplné slovo. Obecně tedy v~češtině tento znak
užíváme tehdy, chceme-li vyjádřit, že jím spojené výrazy tvoří těsný významový
celek. Pomlčka se oproti spojovníku využívá pro oddělování částí projevu,
vyjádření rozsahu, vztahu nebo vyznačení přestávky v~řeči, pro uvození
přímé řeči a~pro vyjádření celého čísla při psaní peněžních částek.
Oddělujeme ji z~obou stran mezerami. Komplikovanější situace nastane pouze
tehdy, když je toto znaménko použito ve funkci výrazů a, až, od, do nebo proti.
Spojovník (-) i~pomlčka (--) bývají často zaměňovány se znaménkem minus ($-$),
to však má též své grafické i~významové odlišnosti.
V~knize~\cite{Pruvodce_tvorbou_dokumentu} je vysvětleno, že znak minus má stejnou
šíři i~umístění jako znak plus. Znak minus se používá ve dvou významech, a to
pro označení záporné hodnoty, nebo pro označení operace odčítání. Sazba se v~obou
případech liší: pro označení záporné hodnoty se znak minus a~následující
operand píše bez mezery, pro psaní minus jako odčítání se však mezera uvádí
z~obou stran tohoto znaménka. Internetové příručka Ústavu pro jazyk
český~\cite{Ustav_pro_jazyk_cesky} však uvádí, že je v~korespondenci dovoleno
znak minus ($-$) nahradit pomlčkou (--).

Podle článku~\cite{Zaklady_typografie:Slezakova} se tato chyba (naznačena na
obrázku~\ref{TODO:}) vyskytuje v~textu kvůli absenci znaku pomlčky na klávesnici.
Místo znaku pomlčky, který je pravděpodobně častěji potřebný při psaní textu,
se na klávesnici vyskytuje právě znak spojovníku. I~když nyní už spousta
textových editorů dokáže automaticky nahradit spojovník za pomlčku, tato náhrada
nemusí být stoprocentní. V~programu {\LaTeX} se spojovník zapíše přímo
z~klávesnice jako \verb|-|,
pomlčku je možno zapsat pomocí dvou spojovníků \verb|--| a~znaménko minus je
zapsáno jako spojovník v~matematickém prostředí \verb|$-$| nebo též \verb|$$-$$|.

\todoimage{width=\linewidth,height=1.7in}{Ukázka špatně použitého spojovníku.}


%#######################    3.3 Chybějící popis kapitoly    #######################
\section{Chybějící popis kapitoly}

I když je kapitola rozdělena na několik podkapitol, musí i samotná kapitola
obsahovat její popis. Blog~\cite{Leany_blog} vysvětluje, , že pokud není uveden
popis mezi kapitolou a~její podkapitolou, působí poté práce nedopracovaně. Tuto
skutečnost lze vidět i~na ukázce v~obrázku~\ref{TODO:}. V~tomto místě
se hodí napsat 1--2 odstavce, kde bude vysvětlené o~čem ta kapitola je, a~co se
v~ní čtenář dozví.
\todoimage{width=\linewidth,height=1.7in}{Ukázka chybějícího textu.}


%#######################    3.4 Nadpisy třetí a větší úrovně v obsahu    #######################
\section{Nadpisy třetí a~větší úrovně v~obsahu}
V~diplomové práci není vhodné uvádět v~obsahu nadpisy třetí či větší úrovně.
Jak je vidět na obrázku~\ref{TODO:}, obsah bude poté nepřehledný a~zbytečně
dlouhý. Samotná třetí úroveň nadpisů je velmi podrobná, ale v~diplomové práci
ji lze použít, když bude nečíslovaná. V~programu {\LaTeX} tohoto lze dosáhnout
příkazem \verb|\subsection*{}|. Nadpisy čtvrté a~větší úrovně už by se v~diplomové
práci neměly vyskytovat vůbec.

\todoimage{width=\linewidth,height=1.7in}{Ukázka obsahu s~nadpisy 3 a~více úrovně.}


%#######################    3.5 Absence vektorové grafiky    #######################
\section{Absence vektorové grafiky}
Při vkládání obrázku do textu se musíme zabývat několika otázkami a~jedna z~nich
je určitě jeho kvalita. Pokud má obrázek moc malé rozlišení
nevypadá v~odborné práci dobře. I~přesto, že se obrázek na displeji zdá
dostatečně kvalitní, při tisku může být daný obrázek \uv{rozkostičkovaný}.
Toto nevhodné použití lze vidět i~na obrázku~\ref{TODO:}.
Tento problém kvalitního rozlišení nám může vyřešit použití vektorového obrázku.
Podle knihy~\cite{Pruvodce_tvorbou_dokumentu} je zásadní výhodou vektorového
obrázku jeho uložení, díky kterému si obrázek ponechá vysokou kvalitu i~v~různém
zvětšení. Ale použití vektorové grafiky není vždy vhodné a~v~některých případech
není ani možné. V~knize je proto uvedeno doporučení použít vektorovou grafiku
(formáty SVG, EPS a~PDF) na schémata a~loga, rastrovou grafiku formátu JPG pro
fotografie a~pro ostatní rastrovou grafiku použít formát PNG.

\todoimage{width=\linewidth,height=1.7in}{Ukázka rastrového a~vektorového obrázku.}


%#######################    3.6 Nepoužívání pevné mezery    #######################
\section{Nepoužívání pevné mezery}
Jako spousta jiných věcí i~psaní mezer má svá pravidla. Jak zmiňuje
článek~\cite{Ctenar_12_2015}, i~v~něčem tak samozřejmém, jako je psaní pouhé
mezery se často chybuje: mezera se musí psát za tečkou (nebo též čárkou), ne před
ní a~píše se vždy jen jedna, data se píšou ve formátu \emph{d.~m. yyyy}
a~čísla se oddělují mezerou po tisících (s~výjimkou letopočtu). Při psaní se může
stát, že mezera spojující znaky nebo čísla, vyjde na konec řádku a~tyto znaky
by se rozdělily. Tento případ lze pozorovat i~na obrázku~\ref{TODO:}.
Pro zamezení takových případů existuje právě pevná (nebo též nedělitelná)
mezera.

Pevná mezera se se zobrazí stejně jako normální mezera, ale na rozdíl od normální
mezery, spojí dohromady příslušné znaky a~zablokuje jejich rozdělení na konci
řádku. Podle pravidel Internetové jazykové příručky~\cite{Ustav_pro_jazyk_cesky}
se má pevná mezera použít v~těchto případech:
\begin{itemize}
    \item ve spojení neslabičných předložek \emph{k, s, v, z} s~následujícím
    slovem, např. \emph{v~obrázku, z~funkce},
    \item ve spojení slabičných předložek \emph{o, u} a~spojek \emph{a, i}
    s~následujícím výrazem, např. \emph{o~kapitole, a~to},
    \item členění čísel, např. $\mathit{2~301~000}$, $\mathit{3,141~592~65}$,
    \item mezi číslem a~značkou, např. \emph{25~\%, \copyright~2008},
    \item mezi číslem a~zkratkou počítaného předmětu nebo písmennou značkou
    jednotek a~měn, např. \emph{24~hod., 100~m, 3~000~Kč, 500~¥},
    \item mezi číslem a~názvem počítaného jevu, např. \emph{obrázek~5, 12~metrů,
    I.~patro},
    \item v~kalendářních datech mezi dnem a~měsícem, rok však lze oddělit, např.
    \emph{3.~5. 2000, 26.~dubna 2023}
    \item v~měřítkách map, plánů a~výkresů, v~poměrech nebo při naznačení dělení,
    např. \emph{4~:~7, 1~:~10~000}, $\mathit{12:2=6}$,
    \item v~telefonních, faxových a~jiných číslech členěných mezerou, např. 
    \emph{+420~603~999~226},
    \item ve složených zkratkách (v~případě nutnosti se doporučuje dělit podle
    dílčích celků), v~ustálených spojeních a~v~různých kódech, např.
    \emph{s.~r.~o., m~n.~m., ISO~690},
    \item mezi zkratkami typu \emph{tj., tzv., tzn.} a~výrazem, který za nimi
    bezprostředně následuje, např. \emph{tzv.~pipeline},
    \item mezi zkratkami rodných jmen a~příjmeními, např. \emph{T.~Milet},
    \item mezi zkratkou titulu nebo hodnosti uváděnou před osobním jménem, např.
    \emph{p.~Macková, Ing.~Novák}.
\end{itemize}

Zapsání pevné mezery závisí na použitém textovém editoru. I~když spousta z~nich
už umí tuto pevnou mezeru automaticky doplnit, nemusí mít tato automatizace
stoprocentní úspěšnost. V~programu {\LaTeX} se pevná mezera zapíše znakem tildy
(\texttildelow) a~v~editoru WORD se zase zapíše kombinací kláves
\emph{Ctrl Shift mezera}.

\todoimage{width=\linewidth,height=1.7in}{Ukázka.}

%#######################    3.7 Použití špatných uvozovek    #######################
\section{Použití špatných uvozovek}
\dummyText
\todoimage{width=\linewidth,height=1.7in}{Ukázka.}

% %#######################    3.8 Špatný odkaz na referenci    #######################
% \section{Špatný odkaz na referenci}
% \dummyText
% \todoimage{width=\linewidth,height=1.7in}{Ukázka.}


%*********************************************************************************




%*********************************************************************************
%                               4 ANOTACE V PDF
%*********************************************************************************
\chapter{PDF soubor}

\dummyText



%#######################    4.1 Formát PDF    #######################
\section{Formát PDF}
Většina uživatelů, kteří zachází s~PDF soubory nepotřebují znát vnitřní složení
PDF dokumentů. Spousta programů a~knihoven, pro vytváření či úpravu PDF dokumentu
dokáže při práci s~tímto souborem dostatečně odstínit od syntaxe PDF formátu.
Avšak pro pokročilejší práci s~PDF dokumenty není na obtíž si zjistit pár
základních informací o~uložení dat v~tomto formátu.
PDF dokument má podobu textového souboru a~na jeho pochopení jsou v~této sekci
vysvětleny jeho 4 základní stavební bloky. Tato sekce čerpá informace ze
standardu PDF~32000-1~\cite{PDF32000-1:2008}.


%---------- 4.1.1 Objekty ----------
\subsection*{Objekty}
\todo{obecný popis}
PDF rozeznává osm typů objektů:
\begin{itemize}
    \item \textbf{Boolean objekt} -- Tyto objekty reprezentují logickou hodnotu.
    Můžou nabýt dvou hodnot, které jsou označeny klíčovými slovy \texttt{true}
    a~\texttt{false}.
    \item \textbf{Číselný objekt} -- Obsahovaná číselná hodnota může být celé,
    nebo reálné číslo. U~zápisu reálných čísel se používá desetinná tečka,
    například \texttt{-3.62}, \texttt{.054}, \texttt{+238.45}. Celá čísla můžou
    být například \texttt{500}, \texttt{+3}, \texttt{-21}.
    \item \textbf{Řetězcový objekt (string)} -- Řetězec se dá zapsat dvěma způsoby,
    a~to jako klasický řetězec, nebo jako řetězec v~hexadecimální podobě. Klasický
    řetězec je zapsán jako posloupnost znaků uzavřená v~kulatých závorkách,
    například \texttt{(Toto je string)}. V~tomto typu řetězce je možné používat
    escape sekvence začínající zpětným lomítkem. Řetězec psaný v~hexadecimální
    podobě lze zapsat jako posloupnost hexadecimálních číslic uzavřenou mezi znaky
    menší než a~větší než, například \texttt{<48656c6c6f>}, \texttt{<776F726C64>}.
    Každá dvojice hexadecimálních číslic tvoří jeden znak zakódovaný v~ASCII
    podobě.
    \item \textbf{Jmenný objekt} -- Objekt jména je sekvence znaků. Znaky, které
    se mohou použít ve jménu jsou takové, které zapadají do rozmezí mezi znakem
    vykřičníku (!) a~znakem tildy (\texttildelow). Ostatní znaky se mohou zapsat jako hexadecimální
    hodnota požadovaného znaku, kterou předchází znak mřížky (\#).
    Jméno musí začínat lomítkem, které se nebere jako jeho součást. Zapsané jméno
    může být například \texttt{/Name}, \texttt{/1.6*xyz}, \texttt{/C\#23}.
    \item \textbf{Objekt pole} -- Objekt typu pole je kolekce, která obsahuje
    objekty. Tyto objekty nemusí být stejného typu -- heterogenní pole.
    Zapisuje se jako prvky pole, které jsou odděleny bílým znakem, uzavřené
    v~hranatých závorkách. Prvkem pole může být objekt pole. Validní pole je
    například \texttt{[(string) -25 [2.75 /Name] true]}.
    \item \textbf{Slovníkový objekt} -- Slovník je kolekce, jejíž prvky jsou
    dvojice objektů. První prvek z~této dvojice se nazývá \emph{klíč} a~vždy to 
    musí být objekt typu jméno. Ve slovníku nesmí existovat více záznamů se
    stejným klíčem. Druhý prvek ze dvojice se nazývá \emph{hodnota}.
    Tento prvek může být objekt jakéhokoli typu. Slovník je uvozen dvojitým znakem
    menší než a~dvojitým znakem větší než, například
    \texttt{<</Key1 2.6 /Key2 /Value2>>}.
    \item \textbf{Objekt datového toku (stream)} -- Stream je sekvence bajtů, 
    která má neomezenou délku. Používá se především pro ukládání velkého množství
    dat, což je například obrázek. Tento objekt se zapisuje jako slovník, za nímž
    následuje klíčové slovo \texttt{stream}, po kterém se musí vyskytovat konec
    řádku. Následují bajty datového toku, které jsou ukončeny koncem řádku
    a~klíčovým slovem \texttt{endstream}. Vyskytovaný slovník nesmí být uveden
    nepřímým odkazem a~musí se v~něm uvádět délka datového toku v~bajtech, pod
    klíčem \texttt{Length}. Každý objekt datového toku musí být zároveň nepřímým
    objektem (vysvětleno později v~této sekci). Validní objekt datového toku je
    například:
\begin{verbatim}
    12 0 obj
    <</Length 20 /Filter /FlateDecode>>
    stream
    xścbd`ŕg`b``8	"y
    DZn
    endstream
    endobj
\end{verbatim}
    \item \textbf{Null objekt} -- Null objekt je speciální objekt, který nabývá
    pouze hodnoty \texttt{null}.
\end{itemize}

Každému objektu se může přiřadit jednoznačný identifikátor, takový objekt se poté
nazývá \textbf{nepřímý objekt}. Na nepřímý objekt potom může být odkazováno
z~jiného objektu, čehož je často využíváno například ve slovníku, kde je uveden
klíč a~hodnota je nepřímý odkaz na objekt. Identifikátor nepřímého objektu má dvě
části. První část je kladné celé číslo, kterému se říká \emph{číslo objektu}.
Druhou částí je tzv. \emph{číslo generace}, které je pro nově generovaný dokument
0. Toto číslo musí být vždy nezáporné celé číslo. Nepřímý objekt se zapíše jako
číslo objektu, poté bílý znak a~číslo generace. Následuje samotný objekt uzavřen
mezi klíčovými slovy \texttt{obj} a~\texttt{endobj}. Validní nepřímý objekt je
například:
\begin{verbatim}
    7 0 obj
        <504446>
    endobj
\end{verbatim}
Nepřímý objekt se dá referencovat pomocí \emph{nepřímého odkazu}. Nepřímý odkaz
se zapíše číslem objektu, číslem generace a~klíčovým slovem \texttt{R}, oddělené
bílými znaky. Odkaz na výše uvedený nepřímý objet se zapíše jako \texttt{7 0 R}.

\todo{Xobject, xref ?}


%---------- 4.1.2 Struktura souboru ----------
\subsection*{Struktura souboru}
\todo{The Four Sections of a PDF}


%---------- 4.1.3 Struktura dokumentu ----------
\subsection*{Struktura dokumentu}
\todo{The Page Tree, Pages, PDF units, Rects and boxes}


%---------- 4.1.4 Content streams ----------
\subsection*{Content streams}


%#######################    4.2 Reprezentace anotací v PDF souboru    #######################
\section{Reprezentace anotací v~PDF souboru}

\DummyText



%#######################    4.3 Grafika v~PDF    #######################
\section{Grafika v~PDF}
\todo{Graphics State}

\cite{PDF32000-1:2008}
\DummyText



%#######################    4.4 Programovací jazyky a knihovny pro zpracování a anotování PDF souborů    #######################
\section{Programovací jazyky a~knihovny pro zpracování a~anotování PDF souborů}

Pro zpracovávání PDF souborů existuje mnoho knihoven v~různých programovacích
jazycích. Výběr programovacího jazyka záleží nejen na požadavcích pro výslednou
aplikaci, ale též na znalostech daného programátora. Samotná knihovna se poté
vybere na základě její funkcionality. 

V~této kapitole jsou popsány různé knihovny, které je možné použít pro zpracování
PDF souborů, jejich speciality a~nedostatky.


%---------- 4.3.1 C# ----------
\subsection*{C\#}

C\# je objektově orientovaný programovací jazyk, vyvinutý firmou Microsoft.
Jazyk C\# je potomkem rodiny jazyků C, je jim tedy podobný a~programátorům těchto
jazyků nebude dlouho trvat se jej naučit. Jazyk C\# je jeden z~nejpoužívanějších
jazyků pro vývoj na platformě .NET.
\cite{CSharp}

Nejznámější C\# knihovna pro práci s~PDF dokumenty je \textbf{iText 7}\footnote{
\href{https://kb.itextpdf.com/home}{https://kb.itextpdf.com/home}
}. Tato knihovna je dostupná pod \emph{Open Source AGPLv3}\footnote{
\href{https://itextpdf.com/how-buy/AGPLv3-license}{https://itextpdf.com/how-buy/AGPLv3-license}
} licencí a~dvěma verzemi komerční licence. 
\todo{popsat funkce knihovny iText 7}

\dummyText


%---------- 4.3.2 JavaScript ----------
\subsection*{JavaScript}

JavaScript je dynamicky typovaný, objektově orientovaný, interpretovaný
programovací jazyk. Nejčastěji se využívá jako skriptovací jazyk používaný
pro vytváření webových stránek, je však často používaný i~mimo prostředí webového
prohlížeče. Nejznámější z~těchto případů je například Node.js, Apache CouchDB
a~Adobe Acrobat. 
\cite{JavaScript}

Pro zpracování PDF souborů v~jazyce JavaScript je možné použít některou
z~následujících knihoven:
\begin{itemize}
    \item \textbf{PDF.js}\footnote{
    \href{http://mozilla.github.io/pdf.js/getting_started/}{http://mozilla.github.io/pdf.js/getting\_started/}
    }\,--\,Tato knihovna byla vyvinuta převážně pro čtení a~vykreslování PDF
    souborů, samotná neumí dané soubory editovat. Jiné knihovny jsou s~touto
    knihovnou často kombinovány pro pokročilejší práci s~PDF soubory. Práce
    s~PDF.js knihovnou závisí na využívání takzvaných Promises, bez kterých nelze
    tuto knihovnu používat, proto je doporučené se s~jejich používáním dobře
    seznámit před jakoukoliv prací s~touto knihovnou. PDF.js je dostupné pod
    licencí \emph{Apache License 2.0}\footnote{
    \href{https://github.com/mozilla/pdf.js/blob/master/LICENSE}{https://github.com/mozilla/pdf.js/blob/master/LICENSE}}.
    
    \item \textbf{pdfAnnotate}\footnote{
    \href{https://github.com/highkite/pdfAnnotate}{https://github.com/highkite/pdfAnnotate}
    }\,--\,Je knihovna vyvinutá specificky pro anotování PDF souborů dostupná pod
    licencí \emph{MIT License}\footnote{
    \href{https://github.com/highkite/pdfAnnotate/blob/master/LICENSE}{https://github.com/highkite/pdfAnnotate/blob/master/LICENSE}
    }. Funguje pouze v~prostředí webového prohlížeče a~Node.js. Samotná knihovna
    neumí číst a~zobrazovat PDF soubory, proto je doporučováno tuto knihovnu
    kombinovat s~výše zmíněnou knihovnou PDF.js. Přidané anotace se zapisují
    na konec PDF souboru a~díky tomu je možné je zobrazit i~mimo vytvořenou
    aplikaci.
    
    \item \textbf{PDF-LIB}\footnote{
    \href{https://pdf-lib.js.org/}{https://pdf-lib.js.org/}
    }\,--\,Tuto knihovnu je možné používat v~jakémkoliv JavaScriptovém prostředí.
    PDF-LIB dokáže vytvářet nové či modifikovat existující PDF soubory. Mezi
    modifikace patří například vytváření a~vyplňování formulářů, vkládání PDF
    stránek a~čtení a~přepisování metadat souboru. Vytváření anotací je možné, ale
    vyžaduje pokročilejší znalost zápisu formátu PDF. Tato knihovna je dostupná
    pod licencí \emph{MIT License}\footnote{
    \href{https://github.com/Hopding/pdf-lib/blob/master/LICENSE.md}{https://github.com/Hopding/pdf-lib/blob/master/LICENSE.md}
    }.
\end{itemize}


%---------- 4.3.3 PHP ----------
\subsection*{PHP}

PHP je skriptovací programovací jazyk, který je především vhodný pro vývoj
dynamických webových stránek. Často je PHP kód vnořený přímo do HTML kódu, kterému
tak přidává dynamičnost. PHP podporuje objektově orientované i~procedurální 
programování. I~když je PHP jazyk používán převážně pro vývoj webu, lze jej využít
i~pro vytvoření aplikace běžící v~příkazové řádce a~pro vývoj
desktopových aplikací.
\cite{PHP_is}, \cite{PHP_can_do}

\begin{itemize}
    \item \textbf{TCPDF}\footnote{
    \href{https://tcpdf.org/}{https://tcpdf.org/}
    }\,--\,Tato knihovna je zaměřena na vytváření PDF dokumentů, mezi její hlavní
    vlastnosti patří podpora fontů, automatické hlavičky a~patičky na stránce,
    dělení slov, zarovnání a~zalamování textu, PDF anotace a~automatické číslování
    stran. TCPDF nepodporuje čtení a~editaci existujících PDF souborů.
    Knihovna TCPDF je dostupná pod \emph{Free Software License}\footnote{
    \href{https://tcpdf.org/docs/license/}{https://tcpdf.org/docs/license/}
    } licencí.

    % \item \textbf{TCPDI}\footnote{
    % \href{https://github.com/pauln/tcpdi}{https://github.com/pauln/tcpdi}
    % }\,--\,\todo{text}
    
    % \dummyShortText[8]

    \item \textbf{FPDI}\footnote{
    \href{https://www.setasign.com/products/fpdi/about/}{https://www.setasign.com/products/fpdi/about/}
    }\,--\,Je knihovna pro používání stránek z~existujícího PDF dokumentu jako
    šablonu do nového PDF dokumentu. Při opakovaném použití jedné šablony tato 
    knihovna zajistí, že šablona bude v~souboru PDF zahrnuta právě jednou. Díky
    této skutečnosti může být výsledné PDF menší velikosti, než PDF vytvořeno jiným
    způsobem. Tato třída je kompatibilní s~výše zmíněnou knihovnou TCPDF. Od verze
    1.6 je knihovna FPDI dostupná pod licencí \emph{MIT license}\footnote{
    \href{https://www.tldrlegal.com/l/mit}{https://www.tldrlegal.com/l/mit}
    }.

\end{itemize}


%---------- 4.3.4 Python ----------
\subsection*{Python}

Python je vysokoúrovňový, interpretovaný programovací jazyk. Má dynamickou
kontrolu datových typů a~podporuje objektově orientované programování. Tyto
skutečnosti z~něj dělají ideální jazyk pro skriptování a~rychlé prototypování
různých aplikací na mnoho platformách. Python je programovací jazyk vhodný i~pro
začátečníky.
\cite{Python}

V~tomto jazyce existuje několik knihoven pro práci s~PDF soubory. Je možné použít
například následující: 
\begin{itemize}
    \item \textbf{PyMuPDF}\footnote{
    \href{https://pymupdf.readthedocs.io/en/latest/}{https://pymupdf.readthedocs.io/en/latest/}
    }\,--\,Je Python verze MuPDF. Je dostupná pod dvěma různými licencemi, a~to pod
    licencí \emph{Open Source\,--\,AGPL}\footnote{
    \href{https://artifex.com/licensing/agpl/}{https://artifex.com/licensing/agpl/}
    } a~komerční\footnote{
    \href{https://artifex.com/licensing/commercial/}{https://artifex.com/licensing/commercial/}
    } licencí. Knihovna se může lišit podle verze s~danou licencí. Tato knihovna
    dokáže například číst a~vyjmout text i~obrázky, číst a~upravovat
    metadata a~vyhledávat text v~existujícím dokumentu.
    Knihovna má podporu pro OCR (Optical Character Recognition), pokud při
    instalaci je nainstalován též Tesseract. Podporované formáty dokumentu jsou
    PDF, XPS, OpenXPS, CBZ, EPUB a~FB2 (eBooks). PyMuPDF umí zacházet
    i~s~populárními formáty obrázků jako jsou PNG, JPEG, BMP, TIFF a~dalšími.
    \todo{vlastnosti pouze pro PDF -- anotovat, vyplňovat formuláře, vytvoření nového}
    \todo{přístup z příkazové řádky}

    \item \textbf{pikepdf}\footnote{
    \href{https://pikepdf.readthedocs.io/en/latest/}{https://pikepdf.readthedocs.io/en/latest/}
    }\,--\,Tuto knihovnu lze používat pro vytváření, čtení i~úpravu PDF dokumentů.
    Je to Python verze C++ knihovny QPDF. Pro požívání knihovny je nutné být
    seznámen se specifikací PDF formátu. Knihovna je dostupná pod
    \emph{Mozilla Public License 2.0}\footnote{
    \href{https://github.com/pikepdf/pikepdf/blob/master/LICENSE.txt}{https://github.com/pikepdf/pikepdf/blob/master/LICENSE.txt}
    } licencí. Pikepdf dokáže kopírovat stránky do jiného PDF souboru, extrahovat
    obrázky, nahradit obrázek za jiný, upravovat metadata souboru a~další.

\end{itemize}


%*********************************************************************************




%*********************************************************************************
%                            5 NÁVRH A IMPLEMENTACE
%*********************************************************************************
%TODO: přejmenovat
\chapter{Návrh a~implementace webové aplikace}

\dummyText


%#######################    5.1 Specifikace požadavků    #######################
\section{Specifikace požadavků}

\dummyText

\dummyText


%#######################    5.2 Využité technologie    #######################
\section{Využité technologie}

\dummyShortText[9]


%---------- 5.2.1 Python ----------
\subsection*{Python}

\dummyText[2]


%---------- 5.2.2 Django ----------
\subsection*{Django}

\dummyShortText[13]

\dummyText


%---------- 5.2.3 HTML, CSS ----------
\subsection*{HTML, CSS}

\dummyText


%---------- 5.2.4 JavaScript ----------
\subsection*{JavaScript}

\dummyText


%#######################    5.3 Program pro vyhledání chyb a jejich následné vyznačení    #######################
\section{Program pro vyhledání chyb a~jejich následné vyznačení}

\DummyText

%---------- 5.3.1 Nalezení okraje stránky ----------
\subsection*{Nalezení okraje stránky}

\dummyShortText[10]

\dummyText[2]

\todoimage{width=\linewidth}{Vyznačení okraje textu.}

%---------- 5.3.2 Nalezení souřadnic vloženého PDF na stránce ----------
\subsection*{Nalezení souřadnic vloženého PDF na stránce}

\dummyShortText[10]

\dummyText[2]



%#######################    5.4 Doplňující program pro použití v~příkazovém řádku    #######################
\section{Doplňující program pro použití v~příkazovém řádku}

\dummyText

\dummyText[2]



%#######################    5.5 Architektura webové aplikace    #######################
\section{Architektura webové aplikace}

\dummyText

\dummyText[2]



%#######################    5.6 Ukázka použití vytvořené webové aplikace    #######################
\section{Ukázka použití vytvořené webové aplikace}

\dummyShortText[13]

\dummyText

\todoimage{width=\linewidth,height=3.3in}{Zvolení PDF a~filtrů na webu.}

\dummyShortText[8]

\todoimage{width=\linewidth,height=3.3in}{Čekání na webu.}

\dummyText

\todoimage{width=\linewidth,height=3.3in}{Anotované PDF na webu.}

%*********************************************************************************




%*********************************************************************************
%                                 6 TESTOVÁNÍ 
%*********************************************************************************
%TODO: přejmenovat
\chapter{Testování a~zhodnocení výsledné aplikace}

\dummyShortText[10]


%#######################    6.1 Ověření správné funkcionality vyhledávání chyb    #######################
\section{Ověření správné funkcionality vyhledávání chyb}


%#######################    6.2 Známé chyby aplikace    #######################
\section{Známé chyby aplikace}


%#######################    6.3 Uživatelské dotazníky    #######################
\section{Uživatelské dotazníky}


%#######################    6.4 Možné budoucí rozšíření aplikace    #######################
\section{Možné budoucí rozšíření aplikace}

%*********************************************************************************




%*********************************************************************************
%                                   7 ZÁVĚR
%*********************************************************************************
\chapter{Závěr}

\dummyShortText[8]

\dummyText[2]
%*********************************************************************************




%===============================================================================

% Pro kompilaci po částech (viz projekt.tex) nutno odkomentovat
%\end{document}

  \fi
  
  % Kompilace po částech (viz výše, nutno odkomentovat a zakomentovat input výše)
  % Compilation piecewise (see above, it is necessary to uncomment it and comment out input above)
  %\subfile{chapters/projekt-01-uvod-introduction}
  % ...
  %\subfile{chapters/projekt-05-zaver-conclusion}

  % Pouzita literatura / Bibliography
  % ----------------------------------------------
\ifslovak
  \makeatletter
  \def\@openbib@code{\addcontentsline{toc}{chapter}{Literatúra}}
  \makeatother
  \bibliographystyle{bib-styles/Pysny/skplain}
\else
  \ifczech
    \makeatletter
    \def\@openbib@code{\addcontentsline{toc}{chapter}{Literatura}}
    \makeatother
    \bibliographystyle{bib-styles/Pysny/czplain}
  \else 
    \makeatletter
    \def\@openbib@code{\addcontentsline{toc}{chapter}{Bibliography}}
    \makeatother
    \bibliographystyle{bib-styles/Pysny/enplain}
  %  \bibliographystyle{alpha}
  \fi
\fi
  \begin{flushleft}
  \bibliography{xmacko13-Kontrola-diplomovych-praci-20-literatura-bibliography}
  \end{flushleft}

  % vynechani stranky v oboustrannem rezimu
  % Skip the page in the two-sided mode
  \iftwoside
    \cleardoublepage
  \fi

  % Prilohy / Appendices
  % ---------------------------------------------
  \appendix
\ifczech
  \renewcommand{\appendixpagename}{Přílohy}
  \renewcommand{\appendixtocname}{Přílohy}
  \renewcommand{\appendixname}{Příloha}
\fi
\ifslovak
  \renewcommand{\appendixpagename}{Prílohy}
  \renewcommand{\appendixtocname}{Prílohy}
  \renewcommand{\appendixname}{Príloha}
\fi
%  \appendixpage

% vynechani stranky v oboustrannem rezimu
% Skip the page in the two-sided mode
%\iftwoside
%  \cleardoublepage
%\fi
  
\ifslovak
%  \section*{Zoznam príloh}
%  \addcontentsline{toc}{section}{Zoznam príloh}
\else
  \ifczech
%    \section*{Seznam příloh}
%    \addcontentsline{toc}{section}{Seznam příloh}
  \else
%    \section*{List of Appendices}
%    \addcontentsline{toc}{section}{List of Appendices}
  \fi
\fi
  \startcontents[chapters]
  \setlength{\parskip}{0pt} 
  % seznam příloh / list of appendices
  % \printcontents[chapters]{l}{0}{\setcounter{tocdepth}{2}}
  
  \ifODSAZ
    \setlength{\parskip}{0.5\bigskipamount}
  \else
    \setlength{\parskip}{0pt}
  \fi
  
  % vynechani stranky v oboustrannem rezimu
  \iftwoside
    \cleardoublepage
  \fi
  
  % Přílohy / Appendices
  \ifenglish
    \input{xmacko13-Kontrola-diplomovych-praci-30-prilohy-appendices-en}
  \else
    % Tento soubor nahraďte vlastním souborem s přílohami (nadpisy níže jsou pouze pro příklad)

% Pro kompilaci po částech (viz projekt.tex), nutno odkomentovat a upravit
%\documentclass[../projekt.tex]{subfiles}
%\begin{document}

% Umístění obsahu paměťového média do příloh je vhodné konzultovat s vedoucím
%\chapter{Obsah přiloženého paměťového média}

%\chapter{Manuál}

%\chapter{Konfigurační soubor}

%\chapter{RelaxNG Schéma konfiguračního souboru}

%\chapter{Plakát}


% Pro kompilaci po částech (viz projekt.tex) nutno odkomentovat
%\end{document}

  \fi
  
  % Kompilace po částech (viz výše, nutno odkomentovat)
  % Compilation piecewise (see above, it is necessary to uncomment it)
  %\subfile{xmacko13-Kontrola-diplomovych-praci-30-prilohy-appendices}
  
\end{document}
