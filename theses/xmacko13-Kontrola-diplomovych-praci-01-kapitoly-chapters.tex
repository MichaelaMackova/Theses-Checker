% Tento soubor nahraďte vlastním souborem s obsahem práce.
%=========================================================================
% Autoři: Michal Bidlo, Bohuslav Křena, Jaroslav Dytrych, Petr Veigend a Adam Herout 2019

% Pro kompilaci po částech (viz projekt.tex), nutno odkomentovat a upravit
%\documentclass[../projekt.tex]{subfiles}
%\begin{document}


\newcommand{\highlight}[1]{\colorbox{purple}{\color{white}#1}}
\newcommand{\todoimage}[2]{
    \begin{figure}[H]
        \centering
        \includegraphics[#1]{obrazky-figures/placeholder.pdf}
        \caption{\textbf{#2} \todo{popisek}}
    \end{figure}
}

\renewcommand{\dummyShortText}[1][1]{}
\renewcommand{\dummyText}[1][1]{}
\renewcommand{\DummyText}{}
\renewcommand{\todoimage}[2]{}


\newcommand{\pdfcode}[3]{

    \noindent\begin{minipage}{\linewidth}
        \lstinputlisting[xleftmargin=.25in, caption=#2, captionpos=b, label=#3]{#1}
    \end{minipage}

}


%TODO: všechny odkazy na knihy rozepsat "z knihy [5]" -> "z knihy Pejsek a Kočička [5]"
%TODO: k odkazům přidat footnote na přesné nalezení + třeba dopsat v jaké kapitole / na jaké stránce
%TODO: odkazy v textu na výpisy


%*********************************************************************************
%                                    1 ÚVOD
%*********************************************************************************
\chapter{Úvod}

\dummyText

\dummyText[2]

\dummyShortText[15]

%*********************************************************************************




%*********************************************************************************
%                                2 ČASTÉ CHYBY
%*********************************************************************************
\chapter{Typografie a~často vyskytované chyby v~diplomových pracích}
U~psaní textu se autor musí řídit nejen gramatickými, ale i~typografickými
pravidly. Toto platí především při psaní odborné práce. \highlight{Větší množství chyb
v~obsahu práce může mít za} \highlight{následek to, že i~kvalitně odvedená práce se bude zdát
neuspokojivá.}

Chyby mohou být způsobeny z~nepozornosti, anebo z~neznalosti, přičemž druhá možnost
je pro autora textu horší, jelikož i~po několikátém přečtení nemusí pisatel vůbec
poznat, že se jedná o~chybu. Správnou volbou textového editoru si tvůrce textu
může usnadnit hledání některých chyb. Několik dnešních textových procesorů poskytuje
alespoň částečnou kontrolu pravopisu, nicméně tato kontrola umí ve spoustě případů
upozornit převážně jen na překlepy. Významové chyby, jako je například záměna slov
\emph{tip, typ} nebo \emph{autorizace, autentizace}, bývají často touto
automatickou kontrolou zanedbávány. Další části této kapitoly popisují několik
chyb, které lze nalézt v~mnoha diplomových pracích a~dále uvádějí, jak se těmto
chybám vyhnout.


%#######################    2.1 Rychlokurz typografie    #######################
\section{Rychlokurz typografie}
\todo{TODO: text}
\DummyText


%#######################    2.2 Přetečení obsahu za okraj    #######################
\section{Přetečení obsahu za okraj}
Přetečení textu za okraj se nejčastěji vyskytuje, když student píše svou diplomovou
práci s~pomocí jazyka {\LaTeX}. Obvykle je to způsobeno tím, že program nedokáže
automaticky zalomit slovo na konci řádku, jak je ukázáno na
obrázku~\ref{pic_overflow}. Toto lze opravit napověděním možného
zalomení nebo přeformulováním věty, kde se daná chyba vyskytuje.
Další typ této chyby je přetečení obrázku za okraj,
který se nestává tak často, ale lze jej udělat v~několika textových editorech.

\begin{figure}[H]
    \label{pic_overflow}
    \centering
    %\includegraphics[width=\linewidth,height=1.7in]{obrazky-figures/placeholder.pdf}
    \includegraphics{obrazky-figures/overflow.pdf}
    \caption{\textbf{Ukázka přetečení za okraj.} \todo{popisek + přetečení obrázku}}
\end{figure}


%#######################    2.3 Spojovník x pomlčka    #######################
\section{Chybné použití spojovníku}
Nesprávné používání spojovníku je chyba, která se vyskytuje nejen v~diplomových
pracích. Spojovník (-) je graficky velmi podobný pomlčce (--), ale významově
se značně liší. Pravidla pro psaní těchto znaků, uvedena v~internetové příručce
Ústavu pro jazyk český~\cite{Ustav_pro_jazyk_cesky},
říkají, že spojovník se píše bez mezer mezi výrazy, které spojuje. Výjimkou
je, naznačuje-li spojovník neúplné slovo. Obecně se tedy v~češtině tento znak
užívá tehdy, \highlight{chce-li autor vyjádřit}, že jím spojené výrazy tvoří těsný významový
celek. Pomlčka se oproti spojovníku využívá pro oddělování částí projevu,
vyjádření rozsahu, vztahu nebo vyznačení přestávky v~řeči, pro uvození
přímé řeči a~pro vyjádření celého čísla při psaní peněžních částek.
Odděluje se z~obou stran mezerami. Komplikovanější situace nastane pouze
tehdy, když je toto znaménko použito ve funkci výrazů a, až, od, do nebo proti.
Spojovník (-) i~pomlčka (--) bývají často zaměňovány se znaménkem minus ($-$),
to však má též své grafické i~významové odlišnosti.
V~knize~\cite{Pruvodce_tvorbou_dokumentu} je vysvětleno, že znak minus má stejnou
šíři i~umístění jako znak plus. Znak minus se používá ve dvou významech, a to
pro označení záporné hodnoty, nebo pro označení operace odčítání. Sazba se v~obou
případech liší: pro označení záporné hodnoty se znak minus a~následující
operand píše bez mezery, pro psaní minus jako odčítání se však mezera uvádí
z~obou stran tohoto znaménka. Internetová příručka Ústavu pro jazyk
český~\cite{Ustav_pro_jazyk_cesky} však uvádí, že je v~korespondenci dovoleno
znak minus ($-$) nahradit pomlčkou (--).

Podle článku~\cite{Zaklady_typografie:Slezakova} se tato chyba (naznačena na
obrázku~\ref{TODO:}) vyskytuje v~textu kvůli absenci znaku pomlčky na klávesnici.
Místo znaku pomlčky, který je při psaní textu pravděpodobně potřebný častěji,
se na klávesnici vyskytuje právě znak spojovníku. I~když nyní už spousta
textových editorů dokáže automaticky nahradit spojovník za pomlčku, tato náhrada
nemusí být stoprocentní. V~programu {\LaTeX} se spojovník zapíše přímo
z~klávesnice jako \verb|-|,
pomlčku je možno zapsat pomocí dvou spojovníků \verb|--| a~znaménko minus je
zapsáno jako spojovník v~matematickém prostředí \verb|$-$| nebo též \verb|$$-$$|.

\todoimage{width=\linewidth,height=1.7in}{Ukázka chybně použitého spojovníku.}


%#######################    2.4 Chybějící popis kapitoly    #######################
\section{Chybějící popis kapitoly}

I když je kapitola rozdělena na několik podkapitol, musí i samotná kapitola
obsahovat \highlight{její popis}. Blog~\cite{Leany_blog} vysvětluje, že pokud není uveden
popis mezi kapitolou a~její podkapitolou, působí poté práce nedopracovaně. Tuto
skutečnost lze vidět i~na ukázce v~obrázku~\ref{TODO:}. V~tomto místě
se hodí napsat 1--2 odstavce, kde bude vysvětlené o~čem daná kapitola je a~co se
v~ní čtenář dozví.
\todoimage{width=\linewidth,height=1.7in}{Ukázka chybějícího textu.}


%#######################    2.5 Nadpisy třetí a větší úrovně v obsahu    #######################
\section{Nadpisy třetí a~větší úrovně v~obsahu}
V~diplomové práci není vhodné v~obsahu uvádět nadpisy třetí či větší úrovně.
Jak je vidět na obrázku~\ref{TODO:}, obsah je poté nepřehledný a~zbytečně
dlouhý. Samotná třetí úroveň nadpisů je velmi podrobná, ale v~diplomové práci
ji lze použít v~případě, když bude nečíslovaná. V~programu {\LaTeX} tohoto lze dosáhnout
příkazem \verb|\subsection*{}|. Nadpisy čtvrté a~větší úrovně už by se v~diplomové
práci neměly vůbec vyskytovat.

\todoimage{width=\linewidth,height=1.7in}{Ukázka obsahu s~nadpisy 3 a~více úrovně.}


%#######################    2.6 Absence vektorové grafiky    #######################
\section{Absence vektorové grafiky}
Při vkládání obrázku do textu se autor musí zabývat několika otázkami a~jedna
z~nich je určitě jeho kvalita. Pokud má obrázek moc malé rozlišení
nevypadá v~odborné práci dobře. I~přesto, že se obrázek na displeji zdá
dostatečně kvalitní, při tisku může být daný obrázek \uv{rozkostičkovaný}.
Toto nevhodné použití lze vidět i~na obrázku~\ref{TODO:}.
Tento problém kvalitního rozlišení nám může vyřešit použití vektorového obrázku.
Podle knihy~\cite{Pruvodce_tvorbou_dokumentu} je zásadní výhodou vektorového
obrázku jeho uložení, díky kterému si obrázek ponechá vysokou kvalitu i~v~různém
zvětšení. Ale použití vektorové grafiky není vždy vhodné a~v~některých případech
není ani možné. V~knize je proto uvedeno doporučení použít vektorovou grafiku
(formáty SVG, EPS a~PDF) na schémata a~loga, rastrovou grafiku formátu JPG pro
fotografie a~pro ostatní rastrovou grafiku použít formát PNG.

\todoimage{width=\linewidth,height=1.7in}{Ukázka rastrového a~vektorového obrázku.}


%#######################    2.7 Nepoužívání pevné mezery    #######################
\section{Nepoužívání pevné mezery}
Jako spousta jiných věcí i~psaní mezer má svá pravidla. Jak zmiňuje
článek~\cite{Ctenar_12_2015}, i~v~něčem tak samozřejmém, jako je psaní pouhé
mezery se často chybuje: mezera se musí psát za tečkou (nebo též čárkou), ne před
ní a~píše se vždy jen jedna, data se píšou ve formátu \emph{d.~m. yyyy}
a~čísla se oddělují mezerou po tisících (s~výjimkou letopočtu). Při psaní se může
stát, že mezera spojující znaky nebo čísla, vyjde na konec řádku a~tyto znaky
by se rozdělily. Tento případ lze pozorovat i~na obrázku~\ref{TODO:}.
Pro zamezení takových případů existuje právě pevná (nebo též nedělitelná)
mezera.

Pevná mezera se se zobrazí stejně jako normální mezera, ale na rozdíl od normální
mezery, spojí dohromady příslušné znaky a~zablokuje jejich rozdělení na konci
řádku. Podle pravidel internetové jazykové příručky~\cite{Ustav_pro_jazyk_cesky}
se má pevná mezera použít v~těchto případech (převzato a~upraveno):
\begin{itemize}
    \item ve spojení neslabičných předložek \emph{k, s, v, z} s~následujícím
    slovem, např. \emph{v~obrázku, z~funkce},
    \item ve spojení slabičných předložek \emph{o, u} a~spojek \emph{a, i}
    s~následujícím výrazem, např. \emph{o~kapitole, a~to},
    \item členění čísel, např. $\mathit{2~301~000}$, $\mathit{3,141~592~65}$,
    \item mezi číslem a~značkou, např. \emph{25~\%, \copyright~2008},
    \item mezi číslem a~zkratkou počítaného předmětu nebo písmennou značkou
    jednotek a~měn, např. \emph{24~hod., 100~m, 3~000~Kč, 500~¥},
    \item mezi číslem a~názvem počítaného jevu, např. \emph{obrázek~5, 12~metrů,
    I.~patro},
    \item v~kalendářních datech mezi dnem a~měsícem, rok však lze oddělit, např.
    \emph{3.~5. 2000, 26.~dubna 2023}
    \item v~měřítkách map, plánů a~výkresů, v~poměrech nebo při naznačení dělení,
    např. \emph{4~:~7, 1~:~10~000}, $\mathit{12:2=6}$,
    \item v~telefonních, faxových a~jiných číslech členěných mezerou, např. 
    \emph{+420~603~999~226},
    \item ve složených zkratkách (v~případě nutnosti se doporučuje dělit podle
    dílčích celků), v~ustálených spojeních a~v~různých kódech, např.
    \emph{s.~r.~o., m~n.~m., ISO~690},
    \item mezi zkratkami typu \emph{tj., tzv., tzn.} a~výrazem, který za nimi
    bezprostředně následuje, např. \emph{tzv.~pipeline},
    \item mezi zkratkami rodných jmen a~příjmeními, např. \emph{T.~Milet},
    \item mezi zkratkou titulu nebo hodnosti uváděnou před osobním jménem, např.
    \emph{p.~Macková, Ing.~Novák}.
\end{itemize}

Zapsání pevné mezery závisí na použitém textovém editoru. I~když spousta z~nich
již umí tuto pevnou mezeru automaticky doplnit, nemusí mít tato automatizace
stoprocentní úspěšnost. V~programu {\LaTeX} se pevná mezera zapíše znakem tildy
(\texttildelow) a~v~editoru WORD se zapíše pomocí kombinace kláves
\emph{Ctrl Shift mezera}.

\todoimage{width=\linewidth,height=1.7in}{Ukázka.}

%#######################    2.8 Použití nesprávných uvozovek    #######################
\section{Použití nesprávných uvozovek}
\dummyText
\todoimage{width=\linewidth,height=1.7in}{Ukázka.}

% %#######################    2.9 Špatný odkaz na referenci    #######################
% \section{Špatný odkaz na referenci}
% \dummyText
% \todoimage{width=\linewidth,height=1.7in}{Ukázka.}


%*********************************************************************************




%*********************************************************************************
%                               3 ANOTACE V PDF
%*********************************************************************************
\chapter{PDF soubor}

\dummyText



%#######################    3.1 Formát PDF    #######################
\section{Formát PDF} \label{format_PDF}
Většina uživatelů, kteří zachází s~PDF soubory nepotřebují znát vnitřní složení
PDF dokumentů. Spousta programů a~knihoven, pro vytváření či úpravu PDF dokumentu
dokáže při práci s~tímto souborem dostatečně odstínit od syntaxe PDF formátu.
Avšak pro pokročilejší práci s~PDF dokumenty není na obtíž si zjistit pár
základních informací o~uložení dat v~tomto formátu.
PDF dokument má podobu textového souboru a~na jeho pochopení jsou v~této sekci
vysvětleny jeho 4 základní stavební bloky. Tato sekce čerpá informace ze
standardu PDF~32000-1~\cite{PDF32000-1:2008}.


%---------- 3.1.1 Objekty ----------
\subsection*{Objekty}
Objekty jsou jedny ze základních stavebních bloků PDF dokumentu. PDF rozeznává
osm typů objektů:
\begin{itemize}
    \item \textbf{Boolean objekt} -- Tyto objekty reprezentují logickou hodnotu.
    Můžou nabýt dvou hodnot, které jsou označeny klíčovými slovy \texttt{true}
    a~\texttt{false}.
    
    \item \textbf{Číselný objekt} -- Obsahovaná číselná hodnota může být celé,
    nebo reálné číslo. U~zápisu reálných čísel se používá desetinná tečka,
    například \texttt{-3.62}, \texttt{.054}, \texttt{+238.45}. Celá čísla můžou
    být například \texttt{500}, \texttt{+3}, \texttt{-21}.
    
    \item \textbf{Řetězcový objekt (string)} -- Řetězec se dá zapsat dvěma způsoby,
    a~to jako klasický řetězec, nebo jako řetězec v~hexadecimální podobě. Klasický
    řetězec je zapsán jako posloupnost znaků uzavřená v~kulatých závorkách,
    například \texttt{(Toto je string)}. V~tomto typu řetězce je možné používat
    escape sekvence začínající zpětným lomítkem. Řetězec psaný v~hexadecimální
    podobě lze zapsat jako posloupnost hexadecimálních číslic uzavřenou mezi znaky
    menší než a~větší než, například \texttt{<48656c6c6f>}, \texttt{<776F726C64>}.
    Každá dvojice hexadecimálních číslic tvoří jeden znak zakódovaný v~ASCII
    podobě.
    
    \item \textbf{Jmenný objekt} -- Objekt jména je sekvence znaků. Znaky, které
    se mohou použít ve jménu jsou takové, které zapadají do rozmezí mezi znakem
    vykřičníku (!) a~znakem tildy (\texttildelow). Ostatní znaky se mohou zapsat
    jako hexadecimální hodnota požadovaného znaku, kterou předchází znak mřížky
    (\#). Jméno musí začínat lomítkem, které se nebere jako jeho součást. Zapsané
    jméno může být například \texttt{/Name}, \texttt{/1.6*xyz}, \texttt{/C\#23}.
    
    \item \textbf{Objekt pole} -- Objekt typu pole je kolekce, která obsahuje
    objekty. Tyto objekty nemusí být stejného typu -- heterogenní pole.
    Zapisuje se jako prvky pole, které jsou odděleny bílým znakem, uzavřené
    v~hranatých závorkách. Prvkem pole může být objekt pole. Validní pole je
    například \texttt{[(string) -25 [2.75 /Name] true]}.
    
    \item \textbf{Slovníkový objekt} -- Slovník je kolekce, jejíž prvky jsou
    dvojice objektů. První prvek z~této dvojice se nazývá \emph{klíč} a~vždy to 
    musí být objekt typu jméno. Ve slovníku nesmí existovat více záznamů se
    stejným klíčem. Druhý prvek ze dvojice se nazývá \emph{hodnota}.
    Tento prvek může být objekt jakéhokoli typu. Slovník je uvozen dvojitým znakem
    menší než a~dvojitým znakem větší než, například
    \texttt{<</Key1 2.6 /Key2 /Value2>>}.

    \item \textbf{Objekt datového toku (stream)} -- Stream je sekvence bajtů, 
    která má neomezenou délku. Používá se především pro ukládání velkého množství
    dat, což je například obrázek. Tento objekt se zapisuje jako slovník, za nímž
    následuje klíčové slovo \texttt{stream}, po kterém se musí vyskytovat konec
    řádku. Následují bajty datového toku, které jsou ukončeny koncem řádku
    a~klíčovým slovem \texttt{endstream}. Vyskytovaný slovník nesmí být uveden
    nepřímým odkazem a~musí se v~něm uvádět délka datového toku v~bajtech, pod
    klíčem \texttt{Length}. Každý objekt datového toku musí být zároveň nepřímým
    objektem (vysvětleno později v~této sekci). Validní objekt datového toku je
    například uveden ve výpise~\ref{code_stream}:
    \pdfcode{code_examples/stream_objects.txt}{TODO:}{code_stream}

    \item \textbf{Null objekt} -- Null objekt je speciální objekt, který nabývá
    pouze hodnoty \texttt{null}.
\end{itemize}

Každému objektu se může přiřadit jednoznačný identifikátor, takový objekt se poté
nazývá \textbf{nepřímý objekt}. Na nepřímý objekt potom může být odkazováno
z~jiného objektu, čehož je často využíváno například ve slovníku, kde je uveden
klíč a~hodnota je nepřímý odkaz na objekt. Identifikátor nepřímého objektu má dvě
části. První část je kladné celé číslo, kterému se říká \emph{číslo objektu}.
Druhou částí je tzv. \emph{číslo generace}, které je pro nově generovaný dokument
0. Toto číslo musí být vždy nezáporné celé číslo. Nepřímý objekt se zapíše jako
číslo objektu, poté bílý znak a~číslo generace. Následuje samotný objekt uzavřen
mezi klíčovými slovy \texttt{obj} a~\texttt{endobj}. Validní nepřímý objekt je
například:
\pdfcode{code_examples/indirect_object.txt}{TODO:}{code_indirect_object}
\noindent Nepřímý objekt se dá referencovat pomocí \emph{nepřímého odkazu}.
Nepřímý odkaz se zapíše číslem objektu, číslem generace a~klíčovým slovem
\texttt{R}, oddělené bílými znaky. Odkaz na výše uvedený nepřímý objet se zapíše
jako \texttt{7 0 R}.


%---------- 3.1.2 Struktura souboru ----------
\subsection*{Struktura souboru}
Tato část popisuje, jak jsou výše popsané objekty uložené v~PDF dokumentu.
Též popisuje, jak je k~nim přistupováno a~jak jsou aktualizovány.
PDF soubor se je rozdělen do 4~částí: 
\begin{itemize}
    \item \textbf{Hlavička} -- Hlavička se vždy vyskytuje na prvním řádku souboru.
    Má tvar komentáře, přesněji \texttt{\%PDF–} a~bezprostředně za tím následuje
    číslo PDF verze, například \texttt{\%PDF–1.7}.
    
    \item \textbf{Tělo} -- Tělo se skládá z~posloupnosti nepřímých objektů. Tyto 
    nepřímé objekty popisují vzhled celého dokumentu (stránky, fonty, obrázky,
    \ldots).
    
    \item \textbf{Tabulka křížových odkazů} -- Tabulka křížových odkazů se používá
    při přístupu k~nepřímým objektům. Tato tabulka obsahuje informaci o~umístění
    každého nepřímého objektu. Tabulka se skládá z~jedné nebo více \emph{sekcí
    křížových odkazů}, která musí začínat řádkem s~klíčovým slovem \texttt{xref}.
    Každá sekce může být rozdělena na několik \emph{podsekcí křížových odkazů}.
    Tyto podsekce vždy začínají řádkem, na kterém se vyskytují dvě čísla. První
    číslo označuje první objekt v~záznamu podsekce a~druhé číslo značí, kolik
    takových záznamů se v~dané podsekci vyskytuje. Pod tímto řádkem se nachází
    samotné záznamy o~nepřímých objektech. Záznam o~využívaném nepřímém objektu
    má tvar \emph{nnnnnnnnnn ggggg \textbf{n}} a~je zakončen koncem řádku.
    Desetimístné číslo \emph{nnnnnnnnnn} označuje offset bajtů od začátku
    dokumentu, po začátek nepřímého objektu. Pětimístné číslo \emph{ggggg} je
    číslo generace. Jedna sekce křížových odkazů, rozdělena na 2~podsekce, celkově
    obsahující 3~záznamy může vypadat například:
    \pdfcode{code_examples/cross_reference_table.txt}{TODO:}{code_cross_reference_table}
    \todo{od verze 1.5 může být ve streamu}
    
    \item \textbf{Patička} -- Patička umožňuje rychlé nalezení tabulky křížových
    odkazů a~jiných speciálních objektů. Proto obecně platí, že by se měl PDF
    soubor číst od konce, kde se vykytuje právě patička. Patička začíná klíčovým
    slovem \texttt{trailer}, za ním následuje slovník patičky a~klíčové slovo
    \texttt{startxref}. Na novém řádku se poté vykytuje číslo, uvádějící počet
    bajtů offsetu od začátku souboru po začátek tabulky křížových odkazů. Jako
    poslední se na samostatném řádku musí objevit výraz \texttt{\%\%EOF}.
    V~celém souboru se může vyskytovat více patiček, to je způsobeno aktualizováním
    daného PDF dokumentu. Na posledním řádku souboru se vždy musí vyskytovat
    výraz \texttt{\%\%EOF}. Patička může vypadat následovně:
    \pdfcode{code_examples/trailer.txt}{TODO:}{code_trailer}
    \todo{od verze 1.5 může být slovník ve streamu}

\end{itemize}

\todoimage{width=\linewidth,height=1.7in}{Ukázka struktury souboru + odkaz.}


%---------- 3.1.3 Struktura dokumentu ----------
\subsection*{Struktura dokumentu}
\highlight{Struktura dokumentu popisuje, jak vypadá vyobrazený dokument.}
Lze si ji představit jako hierarchii objektů vyskytujících se v~těle PDF souboru.
Některé z~důležitých částí tohoto hierarchického stromu jsou:
\begin{itemize}
    \item \textbf{Katalog dokumentu} -- Tento katalog je kořenem celého
    hierarchického stromu. Odkaz na něj lze nalézt ve slovníku patičky, pod klíčem
    \texttt{Root}. Tento katalog obsahuje odkazy na objekty specifikující vzhled
    dokumentu. Objekt katalogu je slovník, ve kterém se musí vyskytovat klíč
    \texttt{Type}, ke kterému je přiřazena hodnota \texttt{/Catalog}. Dalším
    povinným prvkem katalogového slovníku je klíč \texttt{Pages}, jehož hodnota je
    nepřímý odkaz na kořenový objekt stromu stránek. Objekt katalogu dokumentu
    může být například:
    \pdfcode{code_examples/document_catalog.txt}{TODO:}{code_documet_catalog}

    \item \textbf{Strom stránek} -- Strom stránek určuje pořadí zobrazení stránek.
    Listové uzly tohoto stromu jsou typu \emph{objektu stránky}, které mají jiný
    tvar než ostatní uzly tohoto stromu. Uzly stromu stránek jsou typu slovníku,
    ve kterém se musí vyskytovat klíče \texttt{Type}, \texttt{Kids},
    \texttt{Count} a~\texttt{Parent}, jenž není povinný v~kořenovém uzlu.
    Hodnota klíče \texttt{Type} musí v~uzlu stromu stránek být \texttt{/Pages}.
    Ke klíči \texttt{Kids} musí být přiřazena hodnota typu pole, které obsahuje
    nepřímé odkazy na uzly stromu stránek, nebo na objekty stránek. Hodnota klíče
    \texttt{Count} je počet listových uzlů, které jsou potomkem tohoto uzlu. Ke
    klíči \texttt{Parent} je přiřazena hodnota nepřímého odkazu přímého předchůdce
    tohoto uzlu. Uzel stromu stránek může být například:
    \pdfcode{code_examples/page_tree.txt}{TODO:}{code_page_tree}

    \item \textbf{Objekt stránky} -- Objekt stránky je typ listového uzlu stromu
    stránek. Tento uzel má tvar slovníku, ve kterém se musí vyskytovat klíč
    \texttt{Type} s~hodnotou \texttt{/Page}. Dalším povinným záznamem tohoto
    slovníku je klíč \texttt{Parent}, jehož hodnota je nepřímý odkaz na přímého
    předchůdce tohoto listového uzlu stromu stránek. Mezi jiné důležité záznamy
    patří záznamy s~klíčem \texttt{MediaBox}, \texttt{Resources} \highlight{(popsáno dále
    v~této kapitole)}, \texttt{Rotate},
    \texttt{Contents} \highlight{(popsáno dále v~této kapitole)}, \texttt{Annots} aj.
    Jednoduchý objekt stránky může vypadat následovně:
    \pdfcode{code_examples/page_object.txt}{TODO:}{code_page_object}
\end{itemize}


%---------- 3.1.4 Content streams ----------
\subsection*{Content streams} \label{content_streams}
Content stream obsahuje popis vzhledu PDF stránky pomocí instrukcí na její
vykreslení. Tyto instrukce jsou zapsány pomocí PDF objektů, ale na rozdíl od
PDF dokumentu jsou tyto instrukce seřazené a~vykonávají se podle jejich
posloupnosti. \emph{Operand} takové instrukce musí být přímý objekt, který 
nesmí být typu datového toku. Operand může být typ slovník pouze při použití
speciálních operací. \emph{Operátor} instrukce určuje, která akce se provede.
Operátory jsou klíčová slova typu jmenného objektu, kde na rozdíl od PDF dokumentu
se operátory píšou bez počátečního lomítka. Content stream používá pro zapsání
instrukcí postfixovou notaci, tedy nejdříve jsou uvedeny všechny operandy instrukce
a~poté je uveden její operátor.


%---------- 3.1.5 Resources ----------
\subsection*{Resources} \label{resources}
Resources specifikují a~pojmenovávají používané externí objekty z~content streams.
V~content streams se nesmí používat nepřímé odkazy, proto je možné pojmenovat
jednotlivé používané objekty a~definovat je tak jako \emph{named resources}.
Tyto jména lze používat pouze uvnitř content streams a~mimo ně nejsou v~PDF
dokumentu validní. Named resources se používají například pro obrázky a~fonty,
které jsou použity na dané stránce. Objekt pro resources je typu slovníku, který
má několik definovaných klíčů, které je možné použít, např. \texttt{ColorSpace}, 
\texttt{XObject}, \texttt{Font}, \texttt{ProcSet} a~další. Slovník pro resources,
který obsahuje font pod jménem \texttt{F5} a~dva externí objekty pojmenované jako
\texttt{Im1} a~\texttt{Im2}, může vypadat následovně:
\pdfcode{code_examples/resources.txt}{TODO:}{code_resources}
\todo{F5, Im1 a Im2 správné zvýraznění? (emph vs texttt vs none)}



%#######################    3.2 Grafika v~PDF    #######################
\section{Grafika v~PDF}
\todo{Graphics State, PDF units, Rects and boxes, Xobject, xref?}

Tato sekce čerpá informace ze standardu PDF~32000-1~\cite{PDF32000-1:2008}.
Grafika PDF dokumentu je popsána pomocí content streams, jenž je vysvětleno
v~kapitole~\ref{content_streams}. Operátory zde používané spadají do šesti
hlavních skupin:
\begin{itemize}
    \item \textbf{Graphics state operátory} -- Tyto operátory manipulují
    s~datovou strukturou zvanou \emph{graphics state}, která je vysvětlena později
    v~této kapitole.
    \item \textbf{Operátory pro konstrukci křivek} -- Jsou to operátory, které
    specifikují, jak bude vypadat vykreslená křivka. Spadají zde například
    operátory pro vytvoření nové cesty, přidávání zaoblení, přidávání nové části
    křivky a~uzavření tvaru.
    \item \textbf{Operátory pro vymalování křivek} -- Operátory vybarvující
    křivku nebo prostor vyhrazený takovou křivkou.
    \item \textbf{Další vykreslovací operátory} -- Tyto operátory se používají
    pro vykreslení grafických objektů, mezi které patří i~obrázky.
    \item \textbf{Textové operátory} -- Text se v~PDF vykreslí jako několik
    grafických částí, které jsou definovány fontem. Pomocí těchto operátorů
    se specifikují nastavení spojené s~textem a~též zde patří operátory pro
    vykreslení takového textu. 
    \item \textbf{Marked-content operátory} -- Tyto operátory přímo neovlivňují
    vzhled stránky, ale spojují informace a~objekty uvedené v~content
    \highlight{streamu}.
\end{itemize}


%---------- 3.2.1 Souřadnicové systémy ----------
\subsection*{Souřadnicové systémy}
PDF definuje více souřadnicových systémů, ve kterých se definují různé grafické
objekty. Převod mezi těmito souřadnicovými systémy se provádí s~pomocí
transformačních matic, které dokážou otáčet, posunovat nebo měnit měřítko
mapovaného objektu (nemění se přitom samotné data uloženého objektu, jen jak je
zobrazen uvnitř souřadnicového systému). V~PDF se transformační matice značí polem
obsahující šest číselných objektů $[a~b~c~d~e~f]$. Tato transformační matice je
vyobrazena v~rovnici~\eqref{transformation_matrix}.
\begin{equation}\label{transformation_matrix}
    M_T = 
    \begin{pmatrix}
        a & b & 0 \\
        c & d & 0 \\
        d & f & 1
    \end{pmatrix}
\end{equation}
Pokud se potřebuje provést více elementárních transformací 
(např. rotace a~posun), záleží na jejich posloupnosti provedení. Obecně
platí vzorec~\eqref{multiple_matrix_transformations}, kde $M_{T_0}$ značí matici
první provedené transformace, $M_{T_n}$ je matice poslední provedené transformace
a~$M'$ označuje matici, která kombinuje všechny tyto provedené transformace.
\begin{equation} \label{multiple_matrix_transformations}
    M' = M_{T_0} \cdot M_{T_1} \cdots M_{T_{(n-1)}} \cdot M_{T_n}
\end{equation}

Bod $(x, y)$ v~souřadnicovém systému se může matematicky popsat jako vektor
z~rovnice~\eqref{point}.
\begin{equation}\label{point}
    P = 
    \begin{pmatrix}
        x & y & 1
    \end{pmatrix}
\end{equation}
Transformovaný bod (souřadnice bodu po použití všech transformací) se vypočítá
pomocí vzorce~\eqref{transform_point}. $P$ v~tomto vzorci označuje původní vektor
transformovaného bodu, $P'$ je vektor tohoto bodu po transformaci a~$M'$ je matice
kombinující všechny provedené transformace.
\begin{equation} \label{transform_point}
    P' = P \cdot M'
\end{equation}

Dále jsou popsány souřadnicové systémy používané v~PDF. Vztahy mezi těmito
souřadnicovými systémy jsou vyznačeny na obrázku~\ref{coordinate_spaces}.
\begin{itemize}
    \item \textbf{Prostor zařízení (device space)} -- Tento souřadnicový systém je
    závislý na zařízení, kde se zobrazuje vytvořený PDF dokument. Je to poslední
    prostor, do kterého se promítá zobrazení PDF souboru. Zobrazovacím zařízením
    se rozumí například displej obrazovky počítače, papír v~tiskárně, plátno, na
    které promítá projektor aj.
    
    \item \textbf{Uživatelský prostor (user space)} -- Aby se zamezilo nevhodným
    účinkům při zobrazování do prostoru zařízení, definuje PDF vlastní prostor,
    který je nezávislý na zařízení. Tento souřadnicový systém se nazývá
    \emph{uživatelský prostor}. Pro každou stránku PDF dokumentu se tento
    souřadnicový systém uvede do původního stavu. V~uživatelském prostoru je bod
    $(0, 0)$ levý dolní roh, tedy x-ová souřadnice se zvyšuje směrem
    doprava a~y-ová souřadnice roste směrem nahoru. Rozlišení určeno v~jednotkách
    uživatelského prostoru nesouvisí s~rozlišením v~pixelech uvnitř prostoru
    zařízení. Převedení z~uživatelského prostoru do prostoru zařízení se provádí
    pomocí \emph{current transformation matrix (CTM)}. CTM je součástí datové
    struktury \emph{graphics state}, která je popsáná dále v~této kapitole.
    Aplikace zobrazující PDF si tuto CTM matici upraví podle vlastností výstupního
    zařízení, aby se zachovala nezávislost uživatelského prostoru na zařízení. 
    Vyobrazení tohoto jevu jde vidět na obrázku~\ref{pic_user_to_device}.

    \begin{figure}[H]
        \centering
        \label{pic_user_to_device}
        \includegraphics[width=0.5\linewidth]{obrazky-figures/user_to_device_space.pdf}
        \caption{\todo{TODO: popisek} Inspirován obrázkem ze standardu~PDF~32000-1~\cite{PDF32000-1:2008}}
    \end{figure}
    
    \item \textbf{Specializované prostory} -- PDF formát pracuje kromě
    uživatelského prostoru a~prostoru zařízení i~se specializovanými prostory.
    \begin{itemize}
        \item \emph{Textový prostor} -- V~tomto prostoru se definují souřadnice
        textu. Převod z~textového prostoru do uživatelského prostoru se provádí
        pomocí \emph{textové matice} a~několika textových parametrů vyskytujících
        se v~datové struktuře graphics state.

        \item \emph{Glyph space} -- Glyfy znaků fontu musí být definovány v~tomto
        prostoru. Z~glyph space se se poté transformuje do textového prostoru.
        
        \item \emph{Obrázkový prostor} -- Všechny obrázky by měly být definovány
        v~tomto prostoru. Pro správné vykreslení obrázku na stránku se musí dočasně
        upravit CTM.

        \item \emph{Prostor formuláře} -- Formulářový \emph{XObject} (též externí
        objekt, viz dále v~této kapitole) je
        reprezentován jako content stream. Tento XObject je v~jiném content streamu
        brán jako grafický objekt. Prostor, ve kterém je tento formulář definovaný,
        se nazývá prostor formuláře.

        \item \emph{Pattern space} -- PDF formát poznává typ barvy zvaný
        \emph{pattern}. Pattern je definovaný jako content stream a~prostoru, ve
        kterém je definován se říká pattern space.
        
        \item \emph{3D prostor} -- Tento souřadnicový systém je
        trojdimenzionální a~používá se pro vložené 3D díla.
    \end{itemize}
\end{itemize}

\begin{figure}[H]
    \label{coordinate_spaces}
    \includegraphics[width=\linewidth]{obrazky-figures/coordinate_spaces.pdf}
    \caption[Vztah mezi souřadnicovými systémy používaných uvnitř PDF]{Vztah mezi souřadnicovými systémy používaných uvnitř PDF. Každá šipka představuje transformaci mezi dvěma souřadnicovými systémy. Inspirováno obrázkem ze standardu~PDF~32000-1~\cite{PDF32000-1:2008}}
\end{figure}


%---------- 3.2.2 Graphics state ----------
\subsection*{Graphics state}
Pro zobrazení PDF dokumentu se používá datová struktura zvaná \emph{graphics state}.
Každá aplikace zobrazující PDF dokumenty musí umět udržovat tuto datovou strukturu.
Graphics state struktura v~sobě obsahuje několik parametrů, se kterými pracují
operace uvnitř content streamu (popsaný v~kapitole~\ref{content_streams}). Pro
každou stránku se na začátku musí v~graphics state struktuře nastavit výchozí
hodnoty obsahujících parametrů. Tyto parametry jsou rozděleny na dvě skupiny:
závislé na zařízení a~nezávislé na zařízení. Mezi parametry nezávislých na zařízení
patří například CTM (current transformation matrix), barva (aktuální barva
používaná při vykreslovacích operacích), stav textu (devět parametrů popisující
formát vypisovaného textu) a~tloušťka čáry. Parametry závislé na zařízení jsou
například \uv{overprint}, \uv{flatness} a~\uv{smoothness}.

Na jedné stránce se může vyskytovat několik grafických objektů, které se
vykreslují nezávisle na sobě. Pro tyto účely se používá
\emph{graphics state zásobník}, díky kterému je možné dělat lokální úpravy 
datové struktury graphics state. Tento zásobník je LIFO (last in, first out)
a~ukládá se do něj celá datová struktura graphics state. Pro uložení
se musí použít operátor \texttt{q} a~pro odebrání první položky z~vrcholu
zásobníku se musí použít operátor \texttt{Q}, kterým se obnoví posledně uložený
stav datové struktury graphics state. Uvnitř celého content streamu musí být
použit stejný počet operátorů \texttt{q} a~\texttt{Q}.

Pro úpravu parametrů uložených ve struktuře graphics state se používají uvnitř
content streamu specifické operátory. Mezi tyto operátory patří již uvedené
\texttt{q} a~\texttt{Q}, které nepoužívají žádné operandy. Další používané
operátory upravující graphics state jsou například \texttt{cm}, \texttt{w},
\texttt{i} a~jiné. K~operátoru \texttt{cm} se váže šest číselných operandů
\emph{a, b, c, d,} a~\emph{f}. Dohromady tyto operandy specifikují matici
transformace, která bude vynásobena maticí CMT a~následně přiřazena do CMT položky
datové struktury graphics state. Operátor \texttt{w} je unární a~jeho operandem
je číslo \emph{lineWidth}, které specifikuje tloušťku kreslených čar. Unární
operátor \texttt{i} nastaví uvedený číselná operand jako \emph{flatness} parametr
struktury graphics state.



%---------- 3.2.3 Externí objekty ----------
\subsection*{Externí objekty}
Externí objekt (též taky \emph{XObject}) ... \todo{TODO: obecný popis}.
Každý XObject lze vykreslit pomocí operátoru \texttt{Do} vyskytujícího se uvnitř
content streamu. K~tomuto operátoru se váže jeden operand typu jméno. Toto jméno
musí být uvedeno ve slovníku resources (viz kapitola~\ref{resources}) vázaného
na stejnou stránku jako content stream. Přesněji se toto jméno musí objevit
v~položce, která má hodnotu typu slovník, pod klíčem \texttt{XObject}. Operátor
\texttt{Do} má 3 různé chování. Toto chování závisí na hodnotě vázané ke klíči
\texttt{Subtype} uvnitř XObject slovníku, na který se odkazuje operand tohoto
příkazu. Hodnoty klíče \texttt{Subtype} a~k nim vázané chování příkazu
\texttt{Do} jsou:
\begin{itemize}
    \item \texttt{/Image} -- Tyto externí objekty mají ve slovníkové části své
    definice dodatečné prvky. Prvky s~klíčem \texttt{Width} a~\texttt{Height}
    jsou v~těchto objektech povinné. Tyto prvky můžou ovlivnit jak bude
    vykreslený obrázek vypadat, ale podle aktuálního nastavení stránky můžou mít
    omezené možnosti. \todo{Do}
    
    \item \texttt{/Form} -- Takzvaný \emph{Form XObject} je takový, který ve svém
    datovém toku obsahuje content stream (popsán v~sekci~\ref{content_streams}).
    Form XObject může být vykreslen několikrát a~to na různých souřadnicích.
     
    
    \item \texttt{/PS} -- 
\end{itemize}
\todo{popsat funkci \texttt{Do}}


%#######################    3.3 Reprezentace anotací v PDF souboru    #######################
\section{Reprezentace anotací v~PDF souboru}

\DummyText



%#######################    3.4 Programovací jazyky a knihovny pro zpracování a anotování PDF souborů    #######################
\section{Programovací jazyky a~knihovny pro zpracování a~anotování PDF souborů}

Pro zpracovávání PDF souborů existuje mnoho knihoven v~různých programovacích
jazycích. Výběr programovacího jazyka záleží nejen na požadavcích pro výslednou
aplikaci, ale též na znalostech daného programátora. Samotná knihovna se poté
vybere na základě její funkcionality. 

V~této kapitole jsou popsány různé knihovny, které je možné použít pro zpracování
PDF souborů, jejich speciality a~nedostatky.


%---------- 3.3.1 C# ----------
\subsection*{C\#}

C\# je objektově orientovaný programovací jazyk, vyvinutý firmou Microsoft.
Jazyk C\# je potomkem rodiny jazyků C, je jim tedy podobný a~programátorům těchto
jazyků nebude dlouho trvat se jej naučit. Jazyk C\# je jeden z~nejpoužívanějších
jazyků pro vývoj na platformě .NET.
\cite{CSharp}

Nejznámější C\# knihovna pro práci s~PDF dokumenty je \textbf{iText 7}\footnote{
\href{https://kb.itextpdf.com/home}{https://kb.itextpdf.com/home}
}. Tato knihovna je dostupná pod \emph{Open Source AGPLv3}\footnote{
\href{https://itextpdf.com/how-buy/AGPLv3-license}{https://itextpdf.com/how-buy/AGPLv3-license}
} licencí a~dvěma verzemi komerční licence. 
\todo{popsat funkce knihovny iText 7}

\dummyText


%---------- 3.3.2 JavaScript ----------
\subsection*{JavaScript}

JavaScript je dynamicky typovaný, objektově orientovaný, interpretovaný
programovací jazyk. Nejčastěji se využívá jako skriptovací jazyk používaný
pro vytváření webových stránek, je však často používaný i~mimo prostředí webového
prohlížeče. Nejznámější z~těchto případů je například Node.js, Apache CouchDB
a~Adobe Acrobat. 
\cite{JavaScript}

Pro zpracování PDF souborů v~jazyce JavaScript je možné použít některou
z~následujících knihoven:
\begin{itemize}
    \item \textbf{PDF.js}\footnote{
    \href{http://mozilla.github.io/pdf.js/getting_started/}{http://mozilla.github.io/pdf.js/getting\_started/}
    }\,--\,Tato knihovna byla vyvinuta převážně pro čtení a~vykreslování PDF
    souborů, samotná neumí dané soubory editovat. Jiné knihovny jsou s~touto
    knihovnou často kombinovány pro pokročilejší práci s~PDF soubory. Práce
    s~PDF.js knihovnou závisí na využívání takzvaných Promises, bez kterých nelze
    tuto knihovnu používat, proto je doporučené se s~jejich používáním dobře
    seznámit před jakoukoliv prací s~touto knihovnou. PDF.js je dostupné pod
    licencí \emph{Apache License 2.0}\footnote{
    \href{https://github.com/mozilla/pdf.js/blob/master/LICENSE}{https://github.com/mozilla/pdf.js/blob/master/LICENSE}}.
    
    \item \textbf{pdfAnnotate}\footnote{
    \href{https://github.com/highkite/pdfAnnotate}{https://github.com/highkite/pdfAnnotate}
    }\,--\,Je knihovna vyvinutá specificky pro anotování PDF souborů dostupná pod
    licencí \emph{MIT License}\footnote{
    \href{https://github.com/highkite/pdfAnnotate/blob/master/LICENSE}{https://github.com/highkite/pdfAnnotate/blob/master/LICENSE}
    }. Funguje pouze v~prostředí webového prohlížeče a~Node.js. Samotná knihovna
    neumí číst a~zobrazovat PDF soubory, proto je doporučováno tuto knihovnu
    kombinovat s~výše zmíněnou knihovnou PDF.js. Přidané anotace se zapisují
    na konec PDF souboru a~díky tomu je možné je zobrazit i~mimo vytvořenou
    aplikaci.
    
    \item \textbf{PDF-LIB}\footnote{
    \href{https://pdf-lib.js.org/}{https://pdf-lib.js.org/}
    }\,--\,Tuto knihovnu je možné používat v~jakémkoliv JavaScriptovém prostředí.
    PDF-LIB dokáže vytvářet nové či modifikovat existující PDF soubory. Mezi
    modifikace patří například vytváření a~vyplňování formulářů, vkládání PDF
    stránek a~čtení a~přepisování metadat souboru. Vytváření anotací je možné, ale
    vyžaduje pokročilejší znalost zápisu formátu PDF. Tato knihovna je dostupná
    pod licencí \emph{MIT License}\footnote{
    \href{https://github.com/Hopding/pdf-lib/blob/master/LICENSE.md}{https://github.com/Hopding/pdf-lib/blob/master/LICENSE.md}
    }.
\end{itemize}


%---------- 3.3.3 PHP ----------
\subsection*{PHP}

PHP je skriptovací programovací jazyk, který je především vhodný pro vývoj
dynamických webových stránek. Často je PHP kód vnořený přímo do HTML kódu, kterému
tak přidává dynamičnost. PHP podporuje objektově orientované i~procedurální 
programování. I~když je PHP jazyk používán převážně pro vývoj webu, lze jej využít
i~pro vytvoření aplikace běžící v~příkazové řádce a~pro vývoj
desktopových aplikací.
\cite{PHP_is}, \cite{PHP_can_do}

\begin{itemize}
    \item \textbf{TCPDF}\footnote{
    \href{https://tcpdf.org/}{https://tcpdf.org/}
    }\,--\,Tato knihovna je zaměřena na vytváření PDF dokumentů, mezi její hlavní
    vlastnosti patří podpora fontů, automatické hlavičky a~patičky na stránce,
    dělení slov, zarovnání a~zalamování textu, PDF anotace a~automatické číslování
    stran. TCPDF nepodporuje čtení a~editaci existujících PDF souborů.
    Knihovna TCPDF je dostupná pod \emph{Free Software License}\footnote{
    \href{https://tcpdf.org/docs/license/}{https://tcpdf.org/docs/license/}
    } licencí.

    % \item \textbf{TCPDI}\footnote{
    % \href{https://github.com/pauln/tcpdi}{https://github.com/pauln/tcpdi}
    % }\,--\,\todo{text}
    
    % \dummyShortText[8]

    \item \textbf{FPDI}\footnote{
    \href{https://www.setasign.com/products/fpdi/about/}{https://www.setasign.com/products/fpdi/about/}
    }\,--\,Je knihovna pro používání stránek z~existujícího PDF dokumentu jako
    šablonu do nového PDF dokumentu. Při opakovaném použití jedné šablony tato 
    knihovna zajistí, že šablona bude v~souboru PDF zahrnuta právě jednou. Díky
    této skutečnosti může být výsledné PDF menší velikosti než PDF vytvořeno jiným
    způsobem. Tato třída je kompatibilní s~výše zmíněnou knihovnou TCPDF. Od verze
    1.6 je knihovna FPDI dostupná pod licencí \emph{MIT license}\footnote{
    \href{https://www.tldrlegal.com/l/mit}{https://www.tldrlegal.com/l/mit}
    }.

\end{itemize}


%---------- 3.3.4 Python ----------
\subsection*{Python}

Python je vysokoúrovňový, interpretovaný programovací jazyk. Má dynamickou
kontrolu datových typů a~podporuje objektově orientované programování. Tyto
skutečnosti z~něj dělají ideální jazyk pro skriptování a~rychlé prototypování
různých aplikací na mnoho platformách. Python je programovací jazyk vhodný i~pro
začátečníky.
\cite{Python}

V~tomto jazyce existuje několik knihoven pro práci s~PDF soubory. Je možné použít
například následující: 
\begin{itemize}
    \item \textbf{PyMuPDF}\footnote{
    \href{https://pymupdf.readthedocs.io/en/latest/}{https://pymupdf.readthedocs.io/en/latest/}
    }\,--\,Je Python verze MuPDF. Je dostupná pod dvěma různými licencemi, a~to pod
    licencí \emph{Open Source\,--\,AGPL}\footnote{
    \href{https://artifex.com/licensing/agpl/}{https://artifex.com/licensing/agpl/}
    } a~komerční\footnote{
    \href{https://artifex.com/licensing/commercial/}{https://artifex.com/licensing/commercial/}
    } licencí. Knihovna se může lišit podle verze s~danou licencí. Tato knihovna
    dokáže například číst a~vyjmout text i~obrázky, číst a~upravovat
    metadata a~vyhledávat text v~existujícím dokumentu.
    Knihovna má podporu pro OCR (Optical Character Recognition), pokud při
    instalaci je nainstalován též Tesseract. Podporované formáty dokumentu jsou
    PDF, XPS, OpenXPS, CBZ, EPUB a~FB2 (eBooks). PyMuPDF umí zacházet
    i~s~populárními formáty obrázků jako jsou PNG, JPEG, BMP, TIFF a~dalšími.
    \todo{vlastnosti pouze pro PDF -- anotovat, vyplňovat formuláře, vytvoření nového}
    \todo{přístup z příkazové řádky}

    \item \textbf{pikepdf}\footnote{
    \href{https://pikepdf.readthedocs.io/en/latest/}{https://pikepdf.readthedocs.io/en/latest/}
    }\,--\,Tuto knihovnu lze používat pro vytváření, čtení i~úpravu PDF dokumentů.
    Je to Python verze C++ knihovny QPDF. Pro požívání knihovny je nutné být
    seznámen se specifikací PDF formátu. Knihovna je dostupná pod
    \emph{Mozilla Public License 2.0}\footnote{
    \href{https://github.com/pikepdf/pikepdf/blob/master/LICENSE.txt}{https://github.com/pikepdf/pikepdf/blob/master/LICENSE.txt}
    } licencí. Pikepdf dokáže kopírovat stránky do jiného PDF souboru, extrahovat
    obrázky, nahradit obrázek za jiný, upravovat metadata souboru a~další.

\end{itemize}


%*********************************************************************************




%*********************************************************************************
%                            4 NÁVRH A IMPLEMENTACE
%*********************************************************************************
%TODO: přejmenovat
\chapter{Návrh a~implementace aplikace}

\dummyText


%#######################    4.1 Specifikace požadavků    #######################
\section{Specifikace požadavků}

\dummyText

\dummyText


%#######################    4.2 Využité technologie    #######################
\section{Využité technologie}

\dummyShortText[9]


%---------- 4.2.1 Python ----------
\subsection*{Python}

\dummyText[2]


%---------- 4.2.2 Django ----------
\subsection*{Django}

\dummyShortText[13]

\dummyText


%---------- 4.2.3 HTML, CSS ----------
\subsection*{HTML, CSS}

\dummyText


%---------- 4.2.4 JavaScript ----------
\subsection*{JavaScript}

\dummyText


%#######################    4.3 Program pro vyhledání chyb a jejich následné vyznačení    #######################
\section{Program pro vyhledání chyb a~jejich následné vyznačení}

\DummyText

%---------- 4.3.1 Nalezení okraje stránky ----------
\subsection*{Nalezení okraje stránky}

\dummyShortText[10]

\dummyText[2]

\todoimage{width=\linewidth}{Vyznačení okraje textu.}

%---------- 4.3.2 Nalezení souřadnic vloženého PDF na stránce ----------
\subsection*{Nalezení souřadnic vloženého PDF na stránce}

\dummyShortText[10]

\dummyText[2]



%#######################    4.4 Doplňující program pro použití v~příkazovém řádku    #######################
\section{Doplňující program pro použití v~příkazovém řádku}

\dummyText

\dummyText[2]



%#######################    4.5 Architektura webové aplikace    #######################
\section{Architektura webové aplikace}

\dummyText

\dummyText[2]



%#######################    4.6 Ukázka použití vytvořené webové aplikace    #######################
\section{Ukázka použití vytvořené webové aplikace}

\dummyShortText[13]

\dummyText

\todoimage{width=\linewidth,height=3.3in}{Zvolení PDF a~filtrů na webu.}

\dummyShortText[8]

\todoimage{width=\linewidth,height=3.3in}{Čekání na webu.}

\dummyText

\todoimage{width=\linewidth,height=3.3in}{Anotované PDF na webu.}

%*********************************************************************************




%*********************************************************************************
%                                 5 TESTOVÁNÍ 
%*********************************************************************************
%TODO: přejmenovat
\chapter{Testování a~zhodnocení výsledné aplikace}

\dummyShortText[10]


%#######################    5.1 Ověření správné funkcionality vyhledávání chyb    #######################
\section{Ověření správné funkcionality vyhledávání chyb}


%#######################    5.2 Známé chyby aplikace    #######################
\section{Známé chyby aplikace}


%#######################    5.3 Uživatelské dotazníky    #######################
\section{Uživatelské dotazníky}


%#######################    5.4 Možné budoucí rozšíření aplikace    #######################
\section{Možné budoucí rozšíření aplikace}

%*********************************************************************************




%*********************************************************************************
%                                   6 ZÁVĚR
%*********************************************************************************
\chapter{Závěr}

\dummyShortText[8]

\dummyText[2]
%*********************************************************************************




%===============================================================================

% Pro kompilaci po částech (viz projekt.tex) nutno odkomentovat
%\end{document}
