% Tento soubor nahraďte vlastním souborem s obsahem práce.
%=========================================================================
% Autoři: Michal Bidlo, Bohuslav Křena, Jaroslav Dytrych, Petr Veigend a Adam Herout 2019

% Pro kompilaci po částech (viz projekt.tex), nutno odkomentovat a upravit
%\documentclass[../projekt.tex]{subfiles}
%\begin{document}


\newcommand{\highlight}[1]{\colorbox{purple}{\color{white}#1}}
\newcommand{\todoimage}[2]{
    \begin{figure}[H]
        \centering
        \includegraphics[#1]{obrazky-figures/placeholder.pdf}
        \caption{\textbf{#2} \todo{popisek}}
    \end{figure}
}


% \renewcommand{\dummyShortText}[1][1]{}
% \renewcommand{\dummyText}[1][1]{}
% \renewcommand{\DummyText}{}
% \renewcommand{\todoimage}[2]{}


%*********************************************************************************
%                                    1 ÚVOD
%*********************************************************************************
\chapter{Úvod}

\todo{Udělat Úvod}
\dummyText

\dummyText[2]

\dummyShortText[15]

%*********************************************************************************




%*********************************************************************************
%                                2 ČASTÉ CHYBY
%*********************************************************************************
\chapter{Často vyskytované chyby v~diplomových pracích}

\dummyShortText[6]

\dummyText


%#######################    2.1 Přetečení obsahu za okraj    #######################
\section{Přetečení obsahu za okraj}
Tato chyba se nejčastěji vyskytuje, když student píše \highlight{svou}
diplomovou práci s~pomocí jazyka \LaTeX. 

\dummyShortText[10]
\todoimage{width=\linewidth,height=1.7in}{Ukázka přetečení za okraj.}


%#######################    2.2 Spojovník x pomlčka    #######################
\section{Spojovník $\times$ pomlčka}
\dummyShortText[10]
\todoimage{width=\linewidth,height=1.7in}{Ukázka špatně použitého spojovníku.}


%#######################    2.3 Chybějící popis kapitoly    #######################
\section{Chybějící popis kapitoly}
\dummyShortText[10]
\todoimage{width=\linewidth,height=1.7in}{Ukázka chybějícího textu.}


%#######################    2.4 Nadpisy třetí a větší úrovně v obsahu    #######################
\section{Nadpisy třetí a~větší úrovně v~obsahu}
\dummyShortText[10]
\todoimage{width=\linewidth,height=1.7in}{Ukázka obsahu s~nadpisy 3 a~více úrovně.}


%#######################    2.5 Absence vektorové grafiky    #######################
\section{Absence vektorové grafiky}
\dummyShortText[10]
\todoimage{width=\linewidth,height=1.7in}{Ukázka rastrového a~vektorového obrázku.}


%#######################    2.6 Nepoužívání nezalomitelné mezery    #######################
\section{Nepoužívání nezalomitelné mezery}
\dummyText
\todoimage{width=\linewidth,height=1.7in}{Ukázka.}


%*********************************************************************************




%*********************************************************************************
%                               3 ANOTACE V PDF
%*********************************************************************************
\chapter{Anotace v~PDF souboru}

\dummyText


%#######################    3.1 Reprezentace anotací v PDF souboru    #######################
\section{Reprezentace anotací v~PDF souboru}

\DummyText


%#######################    3.2 Programovací jazyky a knihovny pro zpracování a anotování PDF souborů    #######################
\section{Programovací jazyky a~knihovny pro zpracování a~anotování PDF souborů}

Pro zpracovávání PDF souborů existuje mnoho knihoven v~různých programovacích
jazycích. Výběr programovacího jazyka záleží ne jen na požadavcích pro výslednou
aplikaci, ale též na znalostech daného programátora. Samotná knihovna se poté
vybere na základě její funkcionality. 

V~této kapitole jsou popsány různé knihovny, které je možné použít pro zpracování
PDF souborů, jejich speciality a~nedostatky.


%---------- 3.2.1 C# ----------
\subsection*{C\#}

C\# je objektově orientovaný programovací jazyk, vyvinutý firmou Microsoft.
Jazyk C\# je potomkem rodiny jazyků C, je jim tedy podobný a~programátorům těchto
jazyků nebude dlouho trvat se jej naučit. Jazyk C\# je jeden z~nejpoužívanějších
jazyků pro vývoj na platformě .NET.
\cite{CSharp}

\textbf{iText 7}\footnote{\href{https://kb.itextpdf.com/home}{https://kb.itextpdf.com/home}}
\dummyText


%---------- 3.2.2 JavaScript ----------
\subsection*{JavaScript}

JavaScript je dynamicky typovaný, objektově orientovaný, interpretovaný
programovací jazyk. Nejčastěji se využívá jako skriptovací jazyk používaný
pro vytváření webových stránek, je však často používaný i~mimo prostředí webového
prohlížeče. Nejznámější z~těchto případů je například Node.js, Apache CouchDB
a~Adobe Acrobat. 
\cite{JavaScript}

Pro zpracování PDF souborů v~jazyce JavaScript je možné použít některou
z~následujících knihoven:
\begin{itemize}
    \item \textbf{PDF.js}\footnote{
    \href{http://mozilla.github.io/pdf.js/getting_started/}{http://mozilla.github.io/pdf.js/getting\_started/}
    }\,--\,Tato knihovna byla vyvinuta převážně pro čtení a~vykreslování PDF
    souborů, samotná neumí dané soubory editovat. Jiné knihovny jsou s~touto
    knihovnou často kombinovány pro pokročilejší práci s~PDF soubory. Práce
    s~PDF.js knihovnou závisí na využívání takzvaných Promises, bez kterých nelze
    tuto knihovnu používat, proto je doporučené se s~jejich používáním dobře
    seznámit před jakoukoliv prací s~touto knihovnou. PDF.js je dostupné pod
    licencí \emph{Apache License 2.0}\footnote{
    \href{https://github.com/mozilla/pdf.js/blob/master/LICENSE}{https://github.com/mozilla/pdf.js/blob/master/LICENSE}}.
    
    \item \textbf{pdfAnnotate}\footnote{
    \href{https://github.com/highkite/pdfAnnotate}{https://github.com/highkite/pdfAnnotate}
    }\,--\,Je knihovna vyvinutá specificky pro anotování PDF souborů dostupná pod
    licencí \emph{MIT License}\footnote{
    \href{https://github.com/highkite/pdfAnnotate/blob/master/LICENSE}{https://github.com/highkite/pdfAnnotate/blob/master/LICENSE}
    }. Funguje pouze v~prostředí webového prohlížeče a~Node.js. Samotná knihovna
    neumí číst a~zobrazovat PDF soubory, proto je doporučováno tuto knihovnu
    kombinovat s~výše zmíněnou knihovnou PDF.js. Přidané anotace se zapisují
    na konec PDF souboru a~díky tomu je možné je zobrazit i~mimo vytvořenou
    aplikaci.
    
    \item \textbf{PDF-LIB}\footnote{
    \href{https://pdf-lib.js.org/}{https://pdf-lib.js.org/}
    }\,--\,Tuto knihovnu je možné používat v~jakémkoliv JavaScriptovém prostředí.
    PDF-LIB dokáže vytvářet nové či modifikovat existující PDF soubory. Mezi
    modifikace patří například vytváření a~vyplňování formulářů, vkládání PDF
    stránek a~čtení a~přepisování metadat souboru. Vytváření anotací je možné, ale
    vyžaduje pokročilejší znalost zápisu formátu PDF. Tato knihovna je dostupná
    pod licencí \emph{MIT License}\footnote{
    \href{https://github.com/Hopding/pdf-lib/blob/master/LICENSE.md}{https://github.com/Hopding/pdf-lib/blob/master/LICENSE.md}
    }.
\end{itemize}


%---------- 3.2.3 PHP ----------
\subsection*{PHP}

PHP je skriptovací programovací jazyk, který je především vhodný pro vývoj
dynamických webových stránek. \todo{vnořené v HTML}
\cite{PHP}

\dummyShortText[5]


\begin{itemize}
    \item \textbf{TCPDF}\footnote{\href{https://tcpdf.org/}{https://tcpdf.org/}}\,--\,\dummyText
    \item \textbf{TCPDI}\footnote{\href{https://github.com/pauln/tcpdi}{https://github.com/pauln/tcpdi}}\,--\,\dummyShortText[8]
    \item \textbf{FPDI PDF-Parser}\,--\,\dummyShortText[8]
\end{itemize}


%---------- 3.2.4 Python ----------
\subsection*{Python}

Python je vysokoúrovňový, interpretovaný programovací jazyk. Má dynamickou
kontrolu datových typů a~podporuje objektově orientované programování. Tyto
skutečnosti z~něj dělají ideální jazyk pro skriptování a~rychlé prototypování
různých aplikací na mnoho platformách. Python je programovací jazyk vhodný i~pro
začátečníky.
\cite{Python}

V~tomto jazyce existuje několik knihoven pro práci s~PDF soubory. Je možné použít
například následující: 
\begin{itemize}
    \item \textbf{PyMuPDF}\footnote{\href{https://pymupdf.readthedocs.io/en/latest/}{https://pymupdf.readthedocs.io/en/latest/}}\,--\,\dummyText[2]
    \item \textbf{pikepdf}\footnote{\href{https://pikepdf.readthedocs.io/en/latest/}{https://pikepdf.readthedocs.io/en/latest/}}\,--\,\dummyShortText[8]
\end{itemize}


%*********************************************************************************




%*********************************************************************************
%                            4 NÁVRH A IMPLEMENTACE
%*********************************************************************************
%TODO: přejmenovat
\chapter{Návrh a~implementace webové aplikace}

\dummyText


%#######################    4.1 Specifikace požadavků    #######################
\section{Specifikace požadavků}

\dummyText

\dummyText


%#######################    4.2 Využité technologie    #######################
\section{Využité technologie}

\dummyShortText[9]


%---------- 4.2.1 Python ----------
\subsection*{Python}

\dummyText[2]


%---------- 4.2.2 Django ----------
\subsection*{Django}

\dummyShortText[13]

\dummyText


%---------- 4.2.3 HTML, CSS ----------
\subsection*{HTML, CSS}

\dummyText


%---------- 4.2.4 JavaScript ----------
\subsection*{JavaScript}

\dummyText


%#######################    4.3 Program pro vyhledání chyb a jejich následné vyznačení    #######################
\section{Program pro vyhledání chyb a~jejich následné vyznačení}

\DummyText

%---------- 4.3.1 Nalezení okraje stránky ----------
\subsection*{Nalezení okraje stránky}

\dummyShortText[10]

\dummyText[2]

\todoimage{width=\linewidth}{Vyznačení okraje textu.}



%#######################    4.4 Architektura webové aplikace    #######################
\section{Architektura webové aplikace}

\dummyText

\dummyText[2]


%#######################    4.5 Ukázka použití vytvořené webové aplikace    #######################
\section{Ukázka použití vytvořené webové aplikace}

\dummyShortText[13]

\dummyText

\todoimage{width=\linewidth,height=3.3in}{Zvolení PDF a~filtrů na webu.}

\dummyShortText[8]

\todoimage{width=\linewidth,height=3.3in}{Čekání na webu.}

\dummyText

\todoimage{width=\linewidth,height=3.3in}{Anotované PDF na webu.}

%*********************************************************************************




%*********************************************************************************
%                                 5 TESTOVÁNÍ 
%*********************************************************************************
%TODO: přejmenovat
\chapter{Testování a~zhodnocení výsledné aplikace}

\dummyShortText[10]


%#######################    5.1 Ověření správné funkcionality vyhledávání chyb    #######################
\section{Ověření správné funkcionality vyhledávání chyb}


%#######################    5.2 Známé chyby aplikace    #######################
\section{Známé chyby aplikace}


%#######################    5.3 Uživatelské dotazníky    #######################
\section{Uživatelské dotazníky}


%#######################    5.4 Možné budoucí rozšíření aplikace    #######################
\section{Možné budoucí rozšíření aplikace}

%*********************************************************************************




%*********************************************************************************
%                                   6 ZÁVĚR
%*********************************************************************************
\chapter{Závěr}

\dummyShortText[8]

\dummyText[2]
%*********************************************************************************




%===============================================================================

% Pro kompilaci po částech (viz projekt.tex) nutno odkomentovat
%\end{document}
